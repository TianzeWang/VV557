% !TEX program = xelatex
\documentclass{article}
\usepackage{geometry}
\geometry{left = 2.5cm, right = 2.5cm, top = 3cm, bottom = 3cm}
\usepackage[linesnumbered,ruled,longend]{algorithm2e}
\usepackage{amsmath}
\usepackage[makeroom]{cancel}
\usepackage{amsfonts,amssymb}
\usepackage{blkarray}
\usepackage{booktabs}
\usepackage{dsfont}
\usepackage{enumerate}
\usepackage{extarrows}
\usepackage{epsf}
\usepackage{fontspec}
\usepackage{forest}
\usepackage[colorlinks=true,linkcolor=purple]{hyperref}
\usepackage{listings}
\usepackage{mathrsfs}
\usepackage{microtype}
\usepackage{multirow}
\usepackage{setspace}
\usepackage{tikz}
%\usepackage{indentfirst}
%\usepackage[usenames,dvipsnames]{xcolor}
\newfontfamily\Inputmono{Consolas}
\renewcommand\thesection{Exercise 5.\ \arabic{section}}%\arabic{section}}
\renewcommand\thesubsection{\roman{subsection}).}
\renewcommand\thesubsubsection{\roman{subsection}).\alph{subsubsection}.}
\newcommand{\qedhere}{$\hfill\ensuremath{\square}$}
\newcommand{\f}{\frac}
\newcommand\supp{{\rm supp\ }}
\defaultfontfeatures{Mapping=tex-text,Scale=MatchLowercase}
\newcommand\mycommfont[1]{\ttfamily\textcolor{blue}{#1}}
\SetCommentSty{mycommfont}
%\setmainfont{Citadel Script}
%\setmainfont{Chalkboard}
\setmainfont{CMU Bright}
%\setmainfont{Apple Chancery}
\setmonofont{Optima}
\setsansfont{Optima}
%\renewcommand{\familydefault}{\sfdefault}
%\renewcommand{\footnotesize}{\sfdefault}
\setlength{\parskip}{0.25em}
\setlength{\parindent}{0em}

%%%%%%%%%%%Configurations for code%%%%%%%%%%%%%%%%%%%%%%%
\SetKwInOut{Input}{Input}
\SetKwInOut{Output}{Output}
\SetKwProg{Fn}{Function}{\string:}{end}
\SetKwFunction{mstnew}{MST\_New}
\SetKwFunction{tw}{TreeWeight}
\SetKwFunction{dps}{DFS}
\SetKwFunction{con}{Is\_Connected}
\SetKwFunction{hor}{Three\_Fastest\_Horses}
%%%%%%%%%%%Here is the configurations for Code%%%%%%%%%%%

\definecolor{mygreen}{rgb}{0,0.6,0}
\definecolor{mygray}{rgb}{0.7,0.7,0.7}
\definecolor{mymauve}{rgb}{0.58,0,0.82}
\definecolor{mywhite}{rgb}{1,1,1}
\definecolor{myblack}{rgb}{0,0,0}
\definecolor{myblue}{RGB}{27,154,154}
\lstset{
backgroundcolor=\color{white},
basicstyle = \footnotesize\Inputmono,
breakatwhitespace = false,
breaklines = true,
captionpos = b,
commentstyle = \color{mygray}\bfseries,
extendedchars = false,
frame =shadowbox,
framerule=0.5pt,
frameround=tttt,
keepspaces=true,
keywordstyle=\color{myblue}\bfseries, % keyword style
language = Verilog,                     % the language of code
otherkeywords={string},
numbers=left,
numbersep=5pt,
numberstyle=\tiny\color{mymauve},
rulecolor=\color{black},
showspaces=false,
showstringspaces=false,
showtabs=false,
stepnumber=0,
stringstyle=\color{mymauve},        % string literal style
tabsize=2,
title=\lstname
}

%%%%%%%%%%%%%%%%%%%%%%%%%%%%%%%%%%%%%%%%%%%%

\begin{document}
%\setmainfont{Savoye LET}
%\setmainfont{Cormorant Upright}
\setmainfont{Cormorant Upright}
\renewcommand\arraystretch{1.5}


\thispagestyle{empty}

\begin{center}
\begin{large}
\begin{figure}[!htbp]
\centering
\includegraphics[width=0.7\textwidth]{Logo2}
\end{figure}
\hrule
\vspace*{0.25cm}
\sc{ \Large  UM--SJTU Joint Institute \vspace*{0.3em}} \\
\Large  VV557 Methods of Applied Math II\\
\end{large}
\hrulefill

\vspace*{2cm}
\begin{Large}
\sc{{Assignment 5}} \\
\end{Large}
\vspace*{2cm}
\begin{Large}
\sc{{Group 22}}\\
\end{Large}
\vspace*{0.5cm}
\begin{large}
\sc{{Sui, Zijian\ \ 515370910038}} \\
\sc{{Wang, Tianze\ \ 515370910202}} \\
\sc{{Xu, Yisu \ \ 118370910021}} \\
\end{large}
\end{center}
\newpage
\setmainfont{Optima}
\setmonofont{Optima}
\setsansfont{Optima}
%\tableofcontents
%\newpage
\setcounter{page}{1}
\normalsize
\section{}
$L^*$ is the same as $L$ since $a_1 = a_0 = 0$.
\[
    L^* = \frac{d^2}{dx^2}
\]
Green's formula thus becomes
\[
	\int_0^1 (vLu-uL^*v) = \int_0^1 (vu''-uv'')=v(1)u'(1)-u(1)v'(1)-v(0)u'(0)+u(0)v'(0)    
\]    
The set $M$ consists of all functions $u$ s.t.
\[
	u(0)=0
\]
Apply these constraints, the right hand side simplifies to
\[
	v(1)u'(1)-u(1)v'(1)-v(0)u'(0)
\]
where $u'(1),u(1),u(0)$ are arbitrary. The adjoint boundary functionals can then be expressed as
\[
\left\{
\begin{array}{c}
B_1^*v = v(1)=0\\
B_2^*v = v'(1)=0\\
B_3^*v = v(0)=0
\end{array}
\right.
\]
\section{}
\subsection{}
$g(x;\xi)$ should satisfy
\[
	\left\{
	\begin{aligned}
	& L g(x;\xi) = \delta(x-\xi)\\
	& g(0) = g'''(0) = g(1) = g''(1) = 0\\
	\end{aligned}
	\right.
\]
The solution is in the form of 
\[
	g(x;\xi) = H(x-\xi) \cdot \frac{(x-\xi)^3}{6}+ a x^3 + bx^2+cx+d
\]		
where $a,b,c,d$ are real numbers. Plug in the conditions, it will yield to
\[\left\{
	\begin{aligned}
	&u(0) = 0     &\Rightarrow \quad& d = 0\\
	&u'''(0) = 0  &\Rightarrow \quad&a = 0 \\
	&u(1) = 0     &\Rightarrow \quad&\frac{(1-\xi)^3}{6}+b+c = 0\\
	&u''(1) = 0   &\Rightarrow \quad&1-\xi+2b = 0 
	\end{aligned}
  \right.
\]
So 
\[
	g(x;\xi) = H(x-\xi) \cdot \frac{(x-\xi)^3}{6}+ \frac{\xi-1}{2} x^2+\frac{\xi^3-3\xi^2+2}{6}x
\]
\subsection{}
Through \textit{Integral by parts} (Note that here we denote $n^{th}$ order derivative of $u$ as $u^{(n)}$)
\begin{align*}
\int u^{(4)}v &=u^{(3)}v -\int u^{(3)}v' \\
&=u^{(3)}v - u^{(2)}v' + \int u^{(2)}v^{(2)} \\
&=u^{(3)}v - u^{(2)}v' + u'v^{(2)}-\int u'v^{(3)} \\
&=u^{(3)}v - u^{(2)}v' + u'v^{(2)}-uv^{(3)} +\int uv^{(4)} \\
\end{align*}
From the calculation above, now we have 
\[
	L^* = L =  \frac{d^4}{dx^4}
\]	
So Greens' formula is
\[
	\int vLu-uL^*v = u^{(3)}v - u^{(2)}v' + u'v^{(2)}-uv^{(3)}
\]
Plug in the boundaries $0$ and $1$,
\[\begin{aligned}
	\int_0^1 vLu-uL^*v =\quad& u^{(3)}(0)v(0) - u^{(2)}(0)v'(0) + u'(0)v^{(2)}(0)-u(0)v^{(3)}(0)\\& - \left(u^{(3)}(1)v(1) - u^{(2)}(1)v'(1) + u'(1)v^{(2)}(1)-u(1)v^{(3)}(1)\right)
\end{aligned}
\]
With boundary conditions 
\begin{align*}
B_{1} u=u(0), \quad B_{2} u=u^{\prime \prime \prime}(0), \quad B_{3}=u(1), \quad B_{4}=u^{\prime \prime}(1)
\end{align*}
The RHS of green's formula then becomes
\[
	- u^{(2)}(0)v'(0) + u'(0)v^{(2)}(0)-u^{(3)}(1)v(1)-u'(1)v^{(2)}(1)
\]
which is independent of $u$. So the boundary conditions are 
\[
	\left\{
	\begin{aligned}
	&B_1^* v = v'(0) = 0\\
	&B_2^* v = v^{(2)}(0) = 0\\
	&B_3^* v = v(1) = 0\\
	&B_4^* v = v^{(2)}(1) = 0\\
	\end{aligned}
	\right.
\]
With the same strategy, we calculate $v(x) = H(x-\xi) \cdot \frac{(x-\xi)^3}{6}+ a x^3 + bx^2+cx+d$
\[\left\{
	\begin{aligned}
	&v'(0) = 0     &\Rightarrow \quad& c = 0\\
	&v''(0) = 0  &\Rightarrow \quad&b = 0 \\
	&v(1) = 0     &\Rightarrow \quad&\frac{(1-\xi)^3}{6}+a+d = 0\\
	&v''(1) = 0   &\Rightarrow \quad&1-\xi+6a = 0 
	\end{aligned}
  \right.
\]
So the solution is given as 
\[
	g^*(x;\xi) = H(x-\xi) \cdot \frac{(x-\xi)^3}{6}+ \frac{\xi-1}{6} x^3+\frac{\xi^3-3\xi^2+2\xi}{6}
\]
\subsection{}
It is always true for adjoint Green function,
\[
	g^*(x,\xi) = g(\xi,x)
\]
If we want 
\[
	g(x,\xi) = g(\xi,x)
\]
This means $g = g^*$. However, from our previous calculation, it's impossible for $g(x,\xi) = g(\xi,x)$, which proves
\[
	g \neq g^*
\]
\qedhere
\section{}
The fully homogeneous adjoint problem is 
\[\left\{
	\begin{aligned}
	& -v''-v = 0 \quad -\pi < x < \pi\\
	& v(\pi) - v(-\pi) = 0\\
	& v'(\pi) - v'(-\pi) = 0
	\end{aligned}
	\right.
\]
which has a non-trivial solution $v(x)=c$ or $v(x)= c\cdot \sin(x)$ or $v(x) = c\cdot \cos(x)$.
Now that we have 
\begin{align*}
	J(u,v)|_{-\pi}^\pi &= -u'v+uv'|_{-\pi}^\pi \\
	&= -u'(\pi)v(\pi)+u(\pi)v'(\pi)+u'(-\pi)v(-\pi)-u(-\pi)v'(-\pi)\\
	&= [u(\pi)-u(-\pi)]v'(\pi)-[u'(\pi)-u'(-\pi)]v(\pi)+u(-\pi)v'(\pi)-u'(-\pi)v(\pi)+u'(-\pi)v(-\pi)-u(-\pi)v'(-\pi)\\
	&= [u(\pi)-u(-\pi)]v'(\pi)-[u'(\pi)-u'(-\pi)]v(\pi) + [v'(\pi)-v'(-\pi)]u(-\pi)+[v(-\pi)-v(\pi)]u'(-\pi)\\
\end{align*}
Plug in boundary conditions determined by $u$ and $v$,
\begin{align*}
	J(u,v)|_{-\pi}^\pi &=[u(\pi)-u(-\pi)]v'(\pi)-[u'(\pi)-u'(-\pi)]v(\pi) \\
	&=\gamma_1v'(\pi) - \gamma_2v(\pi)\\
	&=B_1uB_2^*v -B_2uB_1^*v
\end{align*}
For solution $v = c\cdot \sin(x)$,
\[
	\int_{-\pi}^\pi f(x) \sin(x) \mathrm{d} x = \gamma_1 \sin'(\pi)- \gamma_2\sin(\pi) = - \gamma_1
\]
For solution $v = c\cdot \cos(x)$,
\[
	\int_{-\pi}^ \pi f(x) \cos(x) \mathrm{d} x = \gamma_1 \cos'(\pi)- \gamma_2\cos(\pi) = \gamma_2
\]
So the conditions are 
\[
	\left.
	\begin{aligned}
	&\int_{-\pi}^\pi f(x) \sin(x) \mathrm{d} x =- \gamma_1\\
	&\int_{-\pi}^ \pi f(x) \cos(x) \mathrm{d} x  = \gamma_2
	\end{aligned}
	\right.
\]
The type of forcing function that can give a periodic solution, i.e.
\[
	\left.
	\begin{aligned}
	&\int_{-\pi}^\pi f(x) \sin(x) \mathrm{d} x =0\\
	&\int_{-\pi}^ \pi f(x) \cos(x) \mathrm{d} x = 0
	\end{aligned}
	\right.
\]
\section{}
We first find the adjoint problem of the original one.
\begin{align*}
	\int_0^1 (u''+\pi^2u)v &= \int_0^1 u''v+\int_0^1 \pi^2uv\\
	&= u'v-uv'+\int_0^1uv''+\int_0^1 \pi^2uv\\
	&= u'v-uv'+\int_0^1(v''+\pi^2 v)u
\end{align*}
So $L^* =\frac{d^{2}}{d x^{2}}+\pi^{2} $. Analyzing the same equation given above,
\begin{align*}
	\int_0^1 vLu-uL^*v dx &= (u'v-uv')|_0^1\\
	&=u'(1)v(1)-u(1)v'(1)-u'(0)v(0)+u(0)v'(0)\\
	&=[u'(0)+u'(1)]v(1)-u'(0)v(1)-[u(0)+u(1)]v'(1)+u(0)v'(1) -u'(0)v(0)+u(0)v'(0)\\
	&=\underbrace{[u'(0)+u'(1)]}_{B_2u}v(1)-\underbrace{[u(0)+u(1)]}_{B_1u}v'(1)-\underbrace{[v(0)+v(1)]}_{B_1^*v}u'(0)+\underbrace{[v'(0)+v'(1)]}_{B_2^* v}u(0)
\end{align*}
So the adjoint problem $M^*$ is given as
\begin{align*}
	&L^*=\frac{d^{2}}{d x^{2}}+\pi^{2} \\
	&B_1^* = v(0)+v(1)\\
	&B_2^* = v'(0)+v'(1)
\end{align*}
Solving this equation, it will lead to two non-trivial solutions:
\begin{align*}
	v_{1} = c_1\cdot \cos(\pi x)\\
	v_{2} = c_2\cdot \sin(\pi x)
\end{align*}
Next we find $w_1$ and $w_2$ s.t.
\begin{align*}
	w_1''+\pi^2 w_1 = L w_1 = v_1 = c_1\cdot \cos(\pi x)\\
	w_2''+\pi^2 w_2 = L w_2 = v_2 = c_2\cdot \sin(\pi x)
\end{align*}
Solving by Mathematica with code
\begin{lstlisting}[language = Mathematica]
DSolve[y''[x] + \[Pi]^2 y[x] == c*Sin[\[Pi] x], y[x], x]
DSolve[y''[x] + \[Pi]^2 y[x] == c*Cos[\[Pi] x], y[x], x]
\end{lstlisting}

\end{document}


	















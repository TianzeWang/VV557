% !TEX program = xelatex
\documentclass{article}
\usepackage{geometry}
\geometry{left = 3cm, right = 3cm, top = 3cm, bottom = 3cm}
\usepackage[linesnumbered,ruled,longend]{algorithm2e}
\usepackage{amsmath}
\usepackage{amsfonts,amssymb}
\usepackage{blkarray}
\usepackage{booktabs}
\usepackage{dsfont}
\usepackage{enumerate}
\usepackage{epsf}
\usepackage{fontspec}
\usepackage{forest}
\usepackage[colorlinks=true,linkcolor=purple]{hyperref}
\usepackage{listings}
\usepackage{mathrsfs}
\usepackage{microtype}
\usepackage{multirow}
\usepackage{setspace}
\usepackage{tikz}
%\usepackage{indentfirst}
%\usepackage[usenames,dvipsnames]{xcolor}
\newfontfamily\Inputmono{Consolas}
\renewcommand\thesection{Exercise 2.\ \arabic{section}}%\arabic{section}}
\renewcommand\thesubsection{\roman{subsection}).}
\renewcommand\thesubsubsection{\roman{subsection}).\alph{subsubsection}.}
\newcommand{\qedhere}{$\hfill\ensuremath{\square}$}
\defaultfontfeatures{Mapping=tex-text,Scale=MatchLowercase}
\newcommand\mycommfont[1]{\ttfamily\textcolor{blue}{#1}}
\SetCommentSty{mycommfont}
%\setmainfont{Citadel Script}
%\setmainfont{Chalkboard}
\setmainfont{CMU Bright}
%\setmainfont{Apple Chancery}
\setmonofont{Optima}
\setsansfont{Optima}
%\renewcommand{\familydefault}{\sfdefault}
%\renewcommand{\footnotesize}{\sfdefault}
\setlength{\parskip}{0.25em}
\setlength{\parindent}{0em}

%%%%%%%%%%%Configurations for code%%%%%%%%%%%%%%%%%%%%%%%
\SetKwInOut{Input}{Input}
\SetKwInOut{Output}{Output}
\SetKwProg{Fn}{Function}{\string:}{end}
\SetKwFunction{mstnew}{MST\_New}
\SetKwFunction{tw}{TreeWeight}
\SetKwFunction{dps}{DFS}
\SetKwFunction{con}{Is\_Connected}
\SetKwFunction{hor}{Three\_Fastest\_Horses}
%%%%%%%%%%%Here is the configurations for Code%%%%%%%%%%%

%\definecolor{mygreen}{rgb}{0,0.6,0}
%\definecolor{mygray}{rgb}{0.7,0.7,0.7}
%\definecolor{mymauve}{rgb}{0.58,0,0.82}
%\definecolor{mywhite}{rgb}{1,1,1}
%\definecolor{myblack}{rgb}{0,0,0}
%\definecolor{myblue}{RGB}{27,154,154}
%\lstset{
% backgroundcolor=\color{white},
% basicstyle = \footnotesize\Inputmono,
% breakatwhitespace = false,
% breaklines = true,
% captionpos = b,
% commentstyle = \color{mygray}\bfseries,
% extendedchars = false,
% frame =shadowbox,
% framerule=0.5pt,
% frameround=tttt,
% keepspaces=true,
% keywordstyle=\color{myblue}\bfseries, % keyword style
% language = Verilog,                     % the language of code
% otherkeywords={string},
% numbers=left,
% numbersep=5pt,
% numberstyle=\tiny\color{mymauve},
% rulecolor=\color{black},
% showspaces=false,
% showstringspaces=false,
% showtabs=false,
% stepnumber=0,
% stringstyle=\color{mymauve},        % string literal style
% tabsize=2,
% title=\lstname
%}

%%%%%%%%%%%%%%%%%%%%%%%%%%%%%%%%%%%%%%%%%%%%

\begin{document}
%\setmainfont{Savoye LET}
%\setmainfont{Cormorant Upright}
\setmainfont{Cormorant Upright}
\renewcommand\arraystretch{1.5}


\thispagestyle{empty}

\begin{center}
\begin{large}
\begin{figure}[!htbp]
\centering
\includegraphics[width=0.7\textwidth]{Logo2}
\end{figure}
\hrule
\vspace*{0.25cm}
\sc{ \Large  UM--SJTU Joint Institute \vspace*{0.3em}} \\
\Large  VV557 Methods of Applied Math II\\
\end{large}
\hrulefill

\vspace*{2cm}
\begin{Large}
\sc{{Assignment 2}} \\
\end{Large}
\vspace*{2cm}
\begin{Large}
\sc{{Group 22}}\\
\end{Large}
\vspace*{0.5cm}
\begin{large}
\sc{{Sui, Zijian\ \ 515370910038}} \\
\sc{{Wang, Tianze\ \ 515370910202}} \\
\sc{{Xu, Yisu \ \ 118370910021}} \\
\end{large}
\end{center}
\newpage
\setmainfont{Optima}
\setmonofont{Optima}
\setsansfont{Optima}
%\tableofcontents
%\newpage
\setcounter{page}{1}
\section{}
\normalsize
\section{}
\section{}
The distribution of $P(\frac{1}{x})$ is given as 
\[
	\mathcal{P}\left(\frac{1}{x}\right)(\varphi) :=\lim _{\varepsilon \rightarrow 0} \int_{|x| > \varepsilon} \frac{\varphi(x)}{x} d x
\]
Observing the distribution of $P(\frac{1}{x^2})$ is given as 
\[
	\mathcal{P}\left(\frac{1}{x^{2}}\right)(\varphi) :=\lim _{\varepsilon \searrow 0} \int_{|x|>\varepsilon} \frac{1}{x^{2}}(\varphi(x)-\varphi(0)) d x
\]
So we express it in the form of 
\[
	\mathcal{P}\left(\frac{1}{x^2}\right)(\varphi) =\lim _{\varepsilon \searrow 0} \int_{|x| > \varepsilon} \left(-\frac{d}{dx}(\frac{1}{x})\right)\cdot \left(\int_\varepsilon^x (\frac{d}{dt}\varphi(t))dt\right) d x
\]
\end{document}























% !TEX program = xelatex
\documentclass{article}
\usepackage{geometry}
\geometry{left = 2.5cm, right = 2.5cm, top = 3cm, bottom = 3cm}
\usepackage[linesnumbered,ruled,longend]{algorithm2e}
\usepackage{amsmath}
\usepackage[makeroom]{cancel}
\usepackage{amsfonts,amssymb}
\usepackage{blkarray}
\usepackage{booktabs}
\usepackage{dsfont}
\usepackage{enumerate}
\usepackage{extarrows}
\usepackage{epsf}
\usepackage{fontspec}
\usepackage{forest}
\usepackage[colorlinks=true,linkcolor=purple]{hyperref}
\usepackage{listings}
\usepackage{mathrsfs}
\usepackage{microtype}
\usepackage{multirow}
\usepackage{setspace}
\usepackage{tikz}
%\usepackage{indentfirst}
%\usepackage[usenames,dvipsnames]{xcolor}
\newfontfamily\Inputmono{Consolas}
\renewcommand\thesection{Exercise 2.\ \arabic{section}}%\arabic{section}}
\renewcommand\thesubsection{\roman{subsection}).}
\renewcommand\thesubsubsection{\roman{subsection}).\alph{subsubsection}.}
\newcommand{\qedhere}{$\hfill\ensuremath{\square}$}
\newcommand\supp{{\rm supp\ }}
\defaultfontfeatures{Mapping=tex-text,Scale=MatchLowercase}
\newcommand\mycommfont[1]{\ttfamily\textcolor{blue}{#1}}
\SetCommentSty{mycommfont}
%\setmainfont{Citadel Script}
%\setmainfont{Chalkboard}
\setmainfont{CMU Bright}
%\setmainfont{Apple Chancery}
\setmonofont{Optima}
\setsansfont{Optima}
%\renewcommand{\familydefault}{\sfdefault}
%\renewcommand{\footnotesize}{\sfdefault}
\setlength{\parskip}{0.25em}
\setlength{\parindent}{0em}

%%%%%%%%%%%Configurations for code%%%%%%%%%%%%%%%%%%%%%%%
\SetKwInOut{Input}{Input}
\SetKwInOut{Output}{Output}
\SetKwProg{Fn}{Function}{\string:}{end}
\SetKwFunction{mstnew}{MST\_New}
\SetKwFunction{tw}{TreeWeight}
\SetKwFunction{dps}{DFS}
\SetKwFunction{con}{Is\_Connected}
\SetKwFunction{hor}{Three\_Fastest\_Horses}
%%%%%%%%%%%Here is the configurations for Code%%%%%%%%%%%

%\definecolor{mygreen}{rgb}{0,0.6,0}
%\definecolor{mygray}{rgb}{0.7,0.7,0.7}
%\definecolor{mymauve}{rgb}{0.58,0,0.82}
%\definecolor{mywhite}{rgb}{1,1,1}
%\definecolor{myblack}{rgb}{0,0,0}
%\definecolor{myblue}{RGB}{27,154,154}
%\lstset{
% backgroundcolor=\color{white},
% basicstyle = \footnotesize\Inputmono,
% breakatwhitespace = false,
% breaklines = true,
% captionpos = b,
% commentstyle = \color{mygray}\bfseries,
% extendedchars = false,
% frame =shadowbox,
% framerule=0.5pt,
% frameround=tttt,
% keepspaces=true,
% keywordstyle=\color{myblue}\bfseries, % keyword style
% language = Verilog,                     % the language of code
% otherkeywords={string},
% numbers=left,
% numbersep=5pt,
% numberstyle=\tiny\color{mymauve},
% rulecolor=\color{black},
% showspaces=false,
% showstringspaces=false,
% showtabs=false,
% stepnumber=0,
% stringstyle=\color{mymauve},        % string literal style
% tabsize=2,
% title=\lstname
%}

%%%%%%%%%%%%%%%%%%%%%%%%%%%%%%%%%%%%%%%%%%%%

\begin{document}
%\setmainfont{Savoye LET}
%\setmainfont{Cormorant Upright}
\setmainfont{Cormorant Upright}
\renewcommand\arraystretch{1.5}


\thispagestyle{empty}

\begin{center}
\begin{large}
\begin{figure}[!htbp]
\centering
\includegraphics[width=0.7\textwidth]{Logo2}
\end{figure}
\hrule
\vspace*{0.25cm}
\sc{ \Large  UM--SJTU Joint Institute \vspace*{0.3em}} \\
\Large  VV557 Methods of Applied Math II\\
\end{large}
\hrulefill

\vspace*{2cm}
\begin{Large}
\sc{{Assignment 2}} \\
\end{Large}
\vspace*{2cm}
\begin{Large}
\sc{{Group 22}}\\
\end{Large}
\vspace*{0.5cm}
\begin{large}
\sc{{Sui, Zijian\ \ 515370910038}} \\
\sc{{Wang, Tianze\ \ 515370910202}} \\
\sc{{Xu, Yisu \ \ 118370910021}} \\
\end{large}
\end{center}
\newpage
\setmainfont{Optima}
\setmonofont{Optima}
\setsansfont{Optima}
%\tableofcontents
%\newpage
\setcounter{page}{1}
\normalsize
\section{}
Since $g \in \mathcal { D } ^ { \prime } \left( \mathbb { R } ^ { 2 } \right), g ( x ) = - \frac { 1 } { 2 \pi } \log | x |$, g(x) is locally integrable, we have $$ \left( \Delta T _ { g } \right) ( \varphi ) = T _ { g } ( \Delta \varphi ) = \lim _ { \varepsilon \rightarrow 0 } \int _ { | x | > \varepsilon } g ( x ) \varphi(x)d x $$
	In polar coordinates, $r=|x|, g ( r ) = - \frac { 1 } { 2 \pi } \log ( r ) , r\geq 0$
	\\According to Green’s second identity:$$\int _ { r > \varepsilon } g ( r ) \Delta \varphi ( r ) d r = \int _ { r > \varepsilon } \varphi ( r ) \Delta g ( r ) d r + \int _ { r = \varepsilon } \left( g \frac { \partial \varphi } { \partial n } - \varphi \frac { \partial g } { \partial n } \right) d \sigma$$
	\\since $\Delta g = \frac { 1 } { r } \frac { \partial g } {\partial r } + \frac { \partial ^ { 2 } g } { \partial ^ { 2 } r } + \frac { 1 } { r ^ { 2 } } \frac { \partial ^ { 2 } g } { \partial \theta } = 0, \int _ { r > \varepsilon } \varphi ( r ) \Delta g ( r ) d r=0 $, 
	$$\int _ { r = \varepsilon } \left( g \frac { \partial \varphi } { \partial n } - \varphi \frac { \partial g } { \partial n } \right) d \sigma=\int _ { r = \varepsilon } \left( \varphi \frac { \partial g } { \partial r } - g \frac { \partial \varphi } { \partial r } \right) d \sigma$$
    $$=\int _ { r = \varepsilon } \left[\left( - \frac { 1 } { 2 \pi } \right) \cdot \frac { 1 } { r } \varphi - \left( - \frac { 1 } { 2 \pi } \right) \cdot \log ( r ) \frac { \partial  \varphi } { \partial r } \right] d \theta$$
	$\varphi \in \mathcal { D } ^ { \prime } \left( \mathbb { R } ^ { 2 } \right)$ implies $\frac { \partial \varphi } { \partial r }$ bounded, so 
	$$ \left| \int _ { r = \varepsilon }  \frac { 1 } { 2 \pi } \cdot log ( r ) \frac { \partial \varphi } { \partial r } d \sigma \right| \leq  constant \cdot \frac { 1 } { 2 \pi }\cdot \log  \varepsilon \cdot 2 \pi \varepsilon \stackrel { \varepsilon \rightarrow 0 } { \longrightarrow } 0$$
    $$ \int _ { r = \varepsilon } \left( - \frac { 1 } { 2 \pi } \right) \frac { 1 } { r } \cdot \varphi d \sigma\stackrel {\varepsilon \rightarrow 0 } { \longrightarrow } - \frac { 1 } { 2 \pi } \cdot \frac { 1 } { \varepsilon } \cdot  \varphi ( x ) \cdot 2 \pi \varepsilon = - \varphi ( 0 )$$
    Therefore, $\Delta T _ { g } \varphi(x)= - \varphi ( 0 ),  \Delta g = -\delta ( x ) $
\section{}
u: $R^2\rightarrow{}R $ is given by u(x,t)=$\left\{\begin{array}{l}\frac{1}{2}\ \
\ t-\left\vert{}x\right\vert{}>0, \\
0\ \ \ otherwise\end{array}\right.$

Now, u(x,t)=$\left\{\begin{array}{l}\frac{1}{2}\ \ \
t>\left\vert{}x\right\vert{}, \\
0\ \ \ t\leq{}\left\vert{}x\right\vert{}\end{array}\right.$ --(0)

So, in any case u(x,t) is a constant function, neither it depends on the values
of t nor x.

Now, differentiate (0) partially with respect to t


\begin{equation}
u_t=\left\{\begin{array}{l}0\ \ \ t>\left\vert{}x\right\vert{}, \\
0\ \ \ t\leq{}\left\vert{}x\right\vert{}\end{array}\right.\
%eq2
\end{equation}

Again differentiate (1) partially with respect to t

=$>$ $u_{tt}$=0 for any t

Now, differentiate (0) partially with respect to x



\begin{equation}
u_x=\left\{\begin{array}{l}0\ \ \ t>\left\vert{}x\right\vert{}, \\
0\ \ \ t\leq{}\left\vert{}x\right\vert{}\end{array}\right.\
%eq3
\end{equation}

Again differentiate (2) partially with respect to x

=$>$ $u_{xx}$=0 for any x

So $u_{tt}-u_{xx}=0-0=0$

\section{}
The distribution of $P(\frac{1}{x})$ is given as 
\[
	\mathcal{P}\left(\frac{1}{x}\right)(\varphi) :=\lim _{\varepsilon \rightarrow 0} \int_{|x| \geq \varepsilon} \frac{\varphi(x)}{x} d x
\]
Observing the distribution of $P(\frac{1}{x^2})$ is given as 
\[
	\mathcal{P}\left(\frac{1}{x^{2}}\right)(\varphi) :=\lim _{\varepsilon \searrow 0} \int_{|x|>\varepsilon} \frac{1}{x^{2}}(\varphi(x)-\varphi(0)) d x
\]
% So we express it in the form of 
% \[
% 	\mathcal{P}\left(\frac{1}{x^2}\right)(\varphi) =\lim _{\varepsilon \searrow 0} \int_{|x| > \varepsilon} \left(-\frac{d}{dx}(\frac{1}{x})\right)\cdot \left(\int_\varepsilon^x (\frac{d}{dt}\varphi(t))dt\right) d x
% \]
According to the definition of weak derivative
\[
	\frac{d}{dx}\mathcal{P}\left(\frac{1}{x}\right)(\varphi) = - \mathcal{P}\left(\frac{1}{x}\right)(\frac{d}{dx}\varphi) 
\]	
We can thus express it into
\begin{align*}
	\frac{d}{dx}\mathcal{P}\left(\frac{1}{x}\right)(\varphi)& = - \lim _{\varepsilon \rightarrow 0} \int_{|x| \geq \varepsilon} \frac{\varphi'(x)}{x} d x \\
	&= -\lim _{\varepsilon \rightarrow 0} (\int_{\varepsilon}^\infty \frac{\varphi'(x)}{x} d x + \int_{-\infty}^{-\varepsilon} \frac{\varphi'(x)}{x} d x)
\end{align*}
Then we investigate on $\displaystyle -\lim _{\varepsilon \rightarrow 0} \int_{-\infty}^{-\varepsilon} \frac{\varphi'(x)}{x} d x$. By applying the basic integral rule $\int u'v = uv + \int uv'$, 
\begin{align*}
	-\lim _{\varepsilon \rightarrow 0} \int_{-\infty}^{-\varepsilon} \frac{\varphi'(x)}{x} d x & = -\lim _{\varepsilon \rightarrow 0}\left(\left.\frac{1}{x}\cdot \varphi(x) \right|_{-\infty}^ {-\varepsilon} + \int_{-\infty}^{-\varepsilon} \frac{\varphi(x)}{x^\prime} d x\right )\\
	&=-\lim _{\varepsilon \rightarrow 0}\left(\left.\frac{1}{x}\cdot \varphi(x) \right|_{-\infty}^ {-\varepsilon} + \int_{-\infty}^{-\varepsilon} \frac{\varphi(x)}{x^2} d x\right ) \\
	&=-\lim _{\varepsilon \rightarrow 0}\left(\left.\frac{1}{x}\cdot \varphi(x) \right|_{-\infty}^ {-\varepsilon} + \int_{-\infty}^{-\varepsilon} \frac{(\varphi(x)-\varphi(0))+ \varphi(0)}{x^2} d x\right )\\
	&=-\lim _{\varepsilon \rightarrow 0}\left(\left.\frac{1}{x}\cdot \varphi(x) \right|_{-\infty}^ {-\varepsilon} + \int_{-\infty}^{-\varepsilon} \frac{(\varphi(x)-\varphi(0))}{x^2} d x+ \int_{-\infty}^{-\varepsilon} \frac{\varphi(0)}{x^2} d x\right )\\
	&=-\lim _{\varepsilon \rightarrow 0}\left( (\frac{\varphi(-\varepsilon)}{-\varepsilon} - 0) + \int_{-\infty}^{-\varepsilon} \frac{(\varphi(x)-\varphi(0))}{x^2} d x+ \int_{-\infty}^{-\varepsilon} \frac{\varphi(0)}{x^2} d x\right )\\
	&=-\lim _{\varepsilon \rightarrow 0}\left( \frac{\varphi(-\varepsilon)}{-\varepsilon} + \int_{-\infty}^{-\varepsilon} \frac{(\varphi(x)-\varphi(0))}{x^2} d x+ \int_{-\infty}^{-\varepsilon} \frac{\varphi(0)}{x^2} d x\right )
\end{align*}
Similarly, we can derive the expression of $-\lim _{\varepsilon \rightarrow 0} \int_{\varepsilon}^{\infty} \frac{\varphi^{\prime}(x)}{x} d x$. Then
\begin{align*}
	\frac{d}{dx}\mathcal{P}\left(\frac{1}{x}\right)(\varphi)& = - \lim _{\varepsilon \rightarrow 0} \int_{|x| \geq \varepsilon} \frac{\varphi'(x)}{x} d x \\
	&= -\lim _{\varepsilon \rightarrow 0} (\int_{\varepsilon}^\infty \frac{\varphi'(x)}{x} d x + \int_{-\infty}^{-\varepsilon} \frac{\varphi'(x)}{x} d x)
\end{align*}
\begin{align*}
	\frac{d}{dx}\mathcal{P}\left(\frac{1}{x}\right)(\varphi)& = - \lim _{\varepsilon \rightarrow 0} \int_{|x| \geq \varepsilon} \frac{\varphi'(x)}{x} d x \\
	&= -\lim _{\varepsilon \rightarrow 0} (\int_{\varepsilon}^\infty \frac{\varphi'(x)}{x} d x + \int_{-\infty}^{-\varepsilon} \frac{\varphi'(x)}{x} d x)\\
	&=-\lim _{\varepsilon \rightarrow 0}\left( \frac{\varphi(-\varepsilon)}{-\varepsilon}+\frac{\varphi(\varepsilon)}{\varepsilon} + \int_{-\infty}^{-\varepsilon} \frac{\varphi(x)-\varphi(0)}{x^2} d x + \int_{\varepsilon}^{\infty} \frac{\varphi(x)-\varphi(0)}{x^{2}} d x+ \int_{-\infty}^{-\varepsilon} \frac{\varphi(0)}{x^2} d x +\int_{\varepsilon}^{\infty} \frac{\varphi(0)}{x^2} d x \right)\\
\end{align*}
Sine $\varphi(x) \in C_0^\infty$, it is continuous at $x = 0$, so 
\[
	\varphi(0^-) = \varphi(0^+)
\]	
which means
\[
	\lim _{\varepsilon \rightarrow 0} \varphi(-\varepsilon) =\lim _{\varepsilon \rightarrow 0}  \varphi(\varepsilon)
\]
Divide by same denominator $\varepsilon$, it yields to 
\[
	\lim _{\varepsilon \rightarrow 0} \frac{\varphi(-\varepsilon)}{\varepsilon} =\lim _{\varepsilon \rightarrow 0}  \frac{\varphi(\varepsilon)}{\varepsilon}
\]
Which means
\[
	\lim _{\varepsilon \rightarrow 0} \left(\frac{\varphi(-\varepsilon)}{-\varepsilon} + \frac{\varphi(\varepsilon)}{\varepsilon}\right) = 0
\]
Also, $\int \frac{1}{x^2} = -\frac{1}{x}$, which is an odd function, and then integral given by
\[
	\lim _{\varepsilon \rightarrow 0} \left( \int_{-\infty}^{-\varepsilon} \frac{\varphi(0)}{x^{2}} d x+\int_{\varepsilon}^{\infty} \frac{\varphi(0)}{x^{2}} d x \right)
\]
is defined on a symmetric interval, which yields to the total integral value to be 0. Then $\frac{d}{dx}\mathcal{P}\left(\frac{1}{x}\right)(\varphi)$ becomes
\[
	\frac{d}{dx}\mathcal{P}\left(\frac{1}{x}\right)(\varphi) = -\lim _{\varepsilon \rightarrow 0} \left(\int_{-\infty}^{-\varepsilon} \frac{\varphi(x)-\varphi(0)}{x^{2}} d x + \int_{\varepsilon}^{\infty} \frac{\varphi(x)-\varphi(0)}{x^{2}} d x \right) = -\lim _{\varepsilon \searrow 0} \int_{|x|>\varepsilon} \frac{1}{x^{2}}(\varphi(x)-\varphi(0)) d x
\]
which means
\[
	\frac{d}{d x} \mathcal{P}\left(\frac{1}{x}\right)=-\mathcal{P}\left(\frac{1}{x^{2}}\right)
\]
\qedhere
\end{document}























% !TEX program = xelatex
\documentclass{article}
\usepackage{geometry}
\geometry{left = 2.5cm, right = 2.5cm, top = 3cm, bottom = 3cm}
\usepackage[linesnumbered,ruled,longend]{algorithm2e}
\usepackage{amsmath}
\usepackage[makeroom]{cancel}
\usepackage{amsfonts,amssymb}
\usepackage{blkarray}
\usepackage{booktabs}
\usepackage{dsfont}
\usepackage{enumerate}
\usepackage{extarrows}
\usepackage{epsf}
\usepackage{fontspec}
\usepackage{forest}
\usepackage[colorlinks=true,linkcolor=purple]{hyperref}
\usepackage{listings}
\usepackage{mathrsfs}
\usepackage{microtype}
\usepackage{multirow}
\usepackage{setspace}
\usepackage{tikz}
%\usepackage{indentfirst}
%\usepackage[usenames,dvipsnames]{xcolor}
\newfontfamily\Inputmono{Consolas}
\renewcommand\thesection{Exercise 3.\ \arabic{section}}%\arabic{section}}
\renewcommand\thesubsection{\roman{subsection}).}
\renewcommand\thesubsubsection{\roman{subsection}).\alph{subsubsection}.}
\newcommand{\qedhere}{$\hfill\ensuremath{\square}$}
\newcommand{\f}{\frac}
\newcommand\supp{{\rm supp\ }}
\defaultfontfeatures{Mapping=tex-text,Scale=MatchLowercase}
\newcommand\mycommfont[1]{\ttfamily\textcolor{blue}{#1}}
\SetCommentSty{mycommfont}
%\setmainfont{Citadel Script}
%\setmainfont{Chalkboard}
\setmainfont{CMU Bright}
%\setmainfont{Apple Chancery}
\setmonofont{Optima}
\setsansfont{Optima}
%\renewcommand{\familydefault}{\sfdefault}
%\renewcommand{\footnotesize}{\sfdefault}
\setlength{\parskip}{0.25em}
\setlength{\parindent}{0em}

%%%%%%%%%%%Configurations for code%%%%%%%%%%%%%%%%%%%%%%%
\SetKwInOut{Input}{Input}
\SetKwInOut{Output}{Output}
\SetKwProg{Fn}{Function}{\string:}{end}
\SetKwFunction{mstnew}{MST\_New}
\SetKwFunction{tw}{TreeWeight}
\SetKwFunction{dps}{DFS}
\SetKwFunction{con}{Is\_Connected}
\SetKwFunction{hor}{Three\_Fastest\_Horses}
%%%%%%%%%%%Here is the configurations for Code%%%%%%%%%%%

%\definecolor{mygreen}{rgb}{0,0.6,0}
%\definecolor{mygray}{rgb}{0.7,0.7,0.7}
%\definecolor{mymauve}{rgb}{0.58,0,0.82}
%\definecolor{mywhite}{rgb}{1,1,1}
%\definecolor{myblack}{rgb}{0,0,0}
%\definecolor{myblue}{RGB}{27,154,154}
%\lstset{
% backgroundcolor=\color{white},
% basicstyle = \footnotesize\Inputmono,
% breakatwhitespace = false,
% breaklines = true,
% captionpos = b,
% commentstyle = \color{mygray}\bfseries,
% extendedchars = false,
% frame =shadowbox,
% framerule=0.5pt,
% frameround=tttt,
% keepspaces=true,
% keywordstyle=\color{myblue}\bfseries, % keyword style
% language = Verilog,                     % the language of code
% otherkeywords={string},
% numbers=left,
% numbersep=5pt,
% numberstyle=\tiny\color{mymauve},
% rulecolor=\color{black},
% showspaces=false,
% showstringspaces=false,
% showtabs=false,
% stepnumber=0,
% stringstyle=\color{mymauve},        % string literal style
% tabsize=2,
% title=\lstname
%}

%%%%%%%%%%%%%%%%%%%%%%%%%%%%%%%%%%%%%%%%%%%%

\begin{document}
%\setmainfont{Savoye LET}
%\setmainfont{Cormorant Upright}
\setmainfont{Cormorant Upright}
\renewcommand\arraystretch{1.5}


\thispagestyle{empty}

\begin{center}
\begin{large}
\begin{figure}[!htbp]
\centering
\includegraphics[width=0.7\textwidth]{Logo2}
\end{figure}
\hrule
\vspace*{0.25cm}
\sc{ \Large  UM--SJTU Joint Institute \vspace*{0.3em}} \\
\Large  VV557 Methods of Applied Math II\\
\end{large}
\hrulefill

\vspace*{2cm}
\begin{Large}
\sc{{Assignment 4}} \\
\end{Large}
\vspace*{2cm}
\begin{Large}
\sc{{Group 22}}\\
\end{Large}
\vspace*{0.5cm}
\begin{large}
\sc{{Sui, Zijian\ \ 515370910038}} \\
\sc{{Wang, Tianze\ \ 515370910202}} \\
\sc{{Xu, Yisu \ \ 118370910021}} \\
\end{large}
\end{center}
\newpage
\setmainfont{Optima}
\setmonofont{Optima}
\setsansfont{Optima}
%\tableofcontents
%\newpage
\setcounter{page}{1}
\normalsize
\section{}
\subsection{}
We know that $\frac{d^4g}{dx^4}=\delta{}\left(x-\xi{}\right)$.

We hence define the candidate $E\left(x,\xi{}\right)=H(x-\xi{})g_{\xi{}}(x)$for
the casual fundamental solution, with initial conditions
$g_{\xi{}}\left(\xi{}\right)=g_{\xi{}}^{'}\left(\xi{}\right)=g_{\xi{}}^{''}\left(\xi{}\right)=0,\
g_{\xi{}}^{(3)}\left(\xi{}\right)=1$.

Assume that $g_{\xi{}}\left(x\right)=ax^3+bx^2+cx+d$, where a, b, c, d are real
numbers.

With initial conditions $\left\{\begin{array}{l}a{\xi{}}^3+b{\xi{}}^2+c\xi{}+d=0
\\
3a{\xi{}}^2+2b\xi{}+c=0 \\
6a\xi{}+2b=0 \\
6a=1\end{array}\right.\ $, we get $\left\{\begin{array}{l}a=\frac{1}{6} \\
b=-\frac{\xi{}}{2} \\
c=\frac{{\xi{}}^2}{2} \\
d=-\frac{{\xi{}}^3}{6}\end{array}\right.$.

So,
$E\left(x,\xi{}\right)=H(x-\xi{})(\frac{1}{6}x^3-\frac{\xi{}}{2}x^2+\frac{{\xi{}}^2}{2}x-\frac{{\xi{}}^3}{6})$.
~\\
\subsection{}
The general solution is
$g_{\xi{}}\left(x,\xi{}\right)=E\left(x,\xi{}\right)+u(x)$, where
$u\left(x\right)=ax^3+bx^2+cx+d$.

So, $g_{\xi{}}\left(x,\xi{}\right)=\left\{\begin{array}{l}ax^3+bx^2+cx+d\
,x<\xi{} \\
\left(a+\frac{1}{6}\right)x^3+\left(b-\frac{\xi{}}{2}\right)x^2+\left(c+\frac{{\xi{}}^2}{2}\right)x+\left(d-\frac{{\xi{}}^3}{6}\right)\
,x>\xi{}\end{array}\right.$

The boundary conditions
$g\left(0,\xi{}\right)=g^{''}\left(0,\xi{}\right)=g\left(1,\xi{}\right)=g^{''}\left(1,\xi{}\right)=0$.

Then we get $\left\{\begin{array}{l}d=0 \\
2b=0 \\
\left(a+\frac{1}{6}\right)+\left(b-\frac{\xi{}}{2}\right)+\left(c+\frac{{\xi{}}^2}{2}\right)+\left(d-\frac{{\xi{}}^3}{6}\right)=0
\\
\left(6a+1\right)+(2b-\xi{})=0\end{array}\right.$.

So, $\left\{\begin{array}{l}a=\frac{\xi{}-1}{6} \\
b=0 \\
c=\frac{{\xi{}}^3}{6}-\frac{{\xi{}}^2}{2}+\frac{\xi{}}{3} \\
d=0\end{array}\right.$

So,
$g_{\xi{}}\left(x,\xi{}\right)=\left\{\begin{array}{l}\frac{\xi{}-1}{6}x^3+(\frac{{\xi{}}^3}{6}-\frac{{\xi{}}^2}{2}+\frac{\xi{}}{3})x\
,x<\xi{} \\
\frac{\xi{}}{6}x^3-\frac{\xi{}}{2}x^2+\left(\frac{{\xi{}}^3}{6}+\frac{\xi{}}{3}\right)x-\frac{{\xi{}}^3}{6},x>\xi{}\end{array}\right.$

\section{}
\subsection{}
We set: $$E ( x , \xi ) = H ( x - \xi ) u _ { \xi } ( x ),$$
	where H is the Heaviside function.
	\\Solve\quad$ L u = 0$:
    $$L y = - y ^ { \prime \prime } - k ^ { 2 } y = 0$$
    $$- \lambda ^ { 2 } - k ^ { 2 } = 0$$
    Solve the equation, we obtain $$\lambda = \pm i k$$
    Thus $y = e ^ { 0 x } \left( C _ { 1 } \cos k x + C _ { 2 } \sin k x \right)=C _ { 1 } \cos k x + C _ { 2 } \sin k x$
    $$L u _ { \xi } = 0,\quad u _ { \xi } ( \xi ) = 0 , \quad u _ { \xi } ^ { \prime } ( \xi ) = -1 $$
    We have, $$
    \left\{ \begin{array} { l } {  C _ { 1 } \cos k \xi + C _ { 2 } \sin k \xi = 0 } \\ { -  C _ { 1 } k \sin k \xi + C _ { 2 } k \cos k \xi = - 1 } \end{array} \right.
    $$
    Solve the equation set, we have$$
    \begin{aligned} C _ { 1 } & = \frac { \sin k \xi } { k } ,\quad  C _ { 2 } = - \frac { \cos k \xi } { k } \\ u _ { \xi } ( x ) & = \frac { \sin k \xi } { k } \cos  k \xi  - \frac { \cos k \xi } { k } \sin k x \end{aligned}
    $$
    Therefore, $$
    \begin{aligned} E ( x , \xi ) & = H ( x - \xi ) u _ { \xi } ( x ) \\ & = H ( x - \xi ) ( \frac { \sin k \xi } { k } \cos k x - \frac { \cos k \xi } { k } \sin k x ) \end{aligned}
    $$
\subsection{}
The general solution of the homogeneous equation $- \frac { d ^ { 2 } u } { d x ^ { 2 } } - k ^ { 2 } u = 0$ is $$u ( x ) =  C _ { 3 }  \cos k x +  C _ { 4 } 
     \sin k x$$
     $$
     g ( x , \xi ) = E ( x , \xi ) + u ( x ) = \left\{ \begin{array} { l } {  C _ { 3 }  \cos k x + C _ { 4 }  \sin k x } \quad { 0 < x < \xi } \\ { \left( \frac { \sin k \xi } { k } +  C _ { 3 }  \right) \cos k x + \left( - \frac { \cos k \xi } { k } +C _ { 4 } \right) \sin k x }  \quad{ \xi < x < 1 } \end{array} \right.
     $$
     We impose the boundary conditions,
     $$
     \left\{ \begin{array}  { l } g(0, \xi ) =  C _ { 3 }  = 0 
     \\ g ( 1 , \xi ) = { \left( \frac { \sin k \xi } { k } +  C _ { 3 }  \right) \cos k + \left( - \frac { \cos k \xi } { k } +C _ { 4 } \right) \sin k }=0 \end{array} \right.
     $$
     Then we obtain,$$
      C _ { 3 }= 0 , C _ { 4 } = \frac { \sin ( k - k \xi ) } { k \sin k }
     $$
     Therefore, $$
     g ( x , \xi ) = \left\{ \begin{array} { l } { \frac { \sin ( k - k \xi ) } { k \sin k } \sin k x }, { 0 < x < \xi } \\ { \frac { \sin k \xi } { k } \cos k x - \frac { \cos k  \sin k \xi } { k \sin k } \sin k x } ,{ \xi < x < 1 } \end{array} \right.$$
\subsection{}
We apply Fourier transform on both sides
\[
	-\widehat{\frac{d^{2} E}{d x^{2}}}-k^{2} \widehat{E}=\widehat{\delta(x-\xi)}
\]
According to Fourier Property
\[
	\widehat{\delta(x-\xi)} (\omega)= \frac{1}{\sqrt{2 \pi}} \int_{-\infty} ^ {+\infty} e^{-i \omega x} \delta(x- \xi) dx = \frac{e^{-i \omega \xi}}{\sqrt{2 \pi}}
\]
Also, we have 
\[
	\widehat{\frac{d^{2} E}{d x^{2}}} = \widehat{D^2 E} = (i \xi)^2 \hat{E} = -\xi^2 \hat E
\]
So the equation then becomes
\[
	\xi^2 \widehat E - k^{2} \widehat{E} = \frac{e^{-i \omega \xi}}{\sqrt{2 \pi}}
\]
Which means
\[
	\widehat{E} = \frac{e^{-i \omega \xi}}{\sqrt{2 \pi} (\omega^2 - k^2)} = \frac{e^{-i \omega \xi}}{\sqrt{2 \pi}} \cdot \frac{1}{2k}(\frac{1}{\omega-k} - \frac{1}{\omega +k}) 
\]
A useful Fourier transformation is (According to the equation $\mathcal{F}D^ \alpha (-ix)^\beta \varphi(x) = (i \xi)^ \alpha D^ \beta \widehat{\varphi(x)} $)
\[
	\mathcal{F}(sgn(x)) (\omega) = \frac{1}{\sqrt{2\pi}} \frac{2}{i \omega}
\]
So
\[
	\mathcal{F}^{-1}\left(\frac{1}{w}\right)=-i \sqrt{\frac{\pi}{2}} \operatorname{sgn}(x)
\]
So the inverse Fourier of $(\frac{1}{\omega-k} - \frac{1}{\omega +k})$ is given as 
\[
	\mathcal{F}^{-1} \left(\frac{1}{\omega-k}- \frac{1}{\omega+k}\right) = (e^{-i k x} - e^{ikx})\cdot (-i \sqrt{\frac{\pi}{2}} sgn(x)) = -2i \sin(kx) sgn(x) \cdot (-i \sqrt{\frac{\pi}{2}} sgn(x))
\]
So the total inverse Fourier transform is 
\[
	E = \mathcal{F}^{-1} (\frac{e^{-i \omega \xi}}{\sqrt{2 \pi}\left(\omega^{2}-k^{2}\right)})= -\frac{\sin (k|x- \xi|)}{2k}
\]
% Recall what we proved in HW3.ii that
% \[
% 	\widehat{e^{-a|x|}}=\frac{a \sqrt{\frac{2}{\pi}}}{a^{2}+\omega^{2}}
% \]
% We apply the reverse Fourier transform, the $e^{-i \omega \xi}$ term lead to a shift according to translation property. So the final equation is given as 
% \[
% 	E =\frac{1}{2i k } e^{-ik|x- \xi|}
% \]
\subsection{}

We have $g = u+E$. Since $u(x)$ satisfies
\[
	-\frac{d^{2} u}{d x^{2}}-k^{2} u=0
\]
We can set 
\[
	u(x) = \alpha \cos(kx) + \beta \sin(kx)
\]
According to two boundary conditions,
\[
\left\{\begin{aligned}
&g(0, \xi)=-\frac{\sin(k \xi)}{2k}+\alpha=0 \\ 
&g(1, \xi)=-\frac{\sin(k (1-\xi))}{2k}+\alpha \cos k+\beta \sin k=0
\end{aligned}
\right.
\]
Synthesis two equations, the answer is given as
\[
	\left\{
	\begin{aligned}
	&\alpha = \frac{\sin(k\xi)}{2k} \\
	&\beta = \frac{\frac{\sin(k(1-\xi))}{2k}-\frac{\sin(k\xi)\cos k}{2k} }{\sin k }
	\end{aligned}
	\right.
\]
So the generation solution is given as 
\[
	g(x) =  -\frac{\sin (k|x- \xi|)}{2k} + \frac{\sin(k\xi)}{2k}\cos(kx)+ \left(\frac{\frac{\sin(k(1-\xi))}{2k}-\frac{\sin(k\xi)\cos k}{2k} }{\sin k }\right)\sin(kx)
\]
\end{document}

	















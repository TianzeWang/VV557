% !TEX program = xelatex
\documentclass{article}
\usepackage{geometry}
\geometry{left = 2.5cm, right = 2.5cm, top = 3cm, bottom = 3cm}
\usepackage[linesnumbered,ruled,longend]{algorithm2e}
\usepackage{amsmath}
\usepackage[makeroom]{cancel}
\usepackage{amsfonts,amssymb}
\usepackage{blkarray}
\usepackage{booktabs}
\usepackage{dsfont}
\usepackage{enumerate}
\usepackage{extarrows}
\usepackage{epsf}
\usepackage{fontspec}
\usepackage{forest}
\usepackage[colorlinks=true,linkcolor=purple]{hyperref}
\usepackage{listings}
\usepackage{mathrsfs}
\usepackage{microtype}
\usepackage{multirow}
\usepackage{setspace}
\usepackage{tikz}
%\usepackage{indentfirst}
%\usepackage[usenames,dvipsnames]{xcolor}
\newfontfamily\Inputmono{Consolas}
\renewcommand\thesection{Exercise 6.\ \arabic{section}}%\arabic{section}}
\renewcommand\thesubsection{\roman{subsection}).}
\renewcommand\thesubsubsection{\roman{subsection}).\alph{subsubsection}.}
\newcommand{\qedhere}{$\hfill\ensuremath{\square}$}
\newcommand{\f}{\frac}
\newcommand\supp{{\rm supp\ }}
\defaultfontfeatures{Mapping=tex-text,Scale=MatchLowercase}
\newcommand\mycommfont[1]{\ttfamily\textcolor{blue}{#1}}
\SetCommentSty{mycommfont}
%\setmainfont{Citadel Script}
%\setmainfont{Chalkboard}
\setmainfont{CMU Bright}
%\setmainfont{Apple Chancery}
\setmonofont{Optima}
\setsansfont{Optima}
%\renewcommand{\familydefault}{\sfdefault}
%\renewcommand{\footnotesize}{\sfdefault}
\setlength{\parskip}{0.25em}
\setlength{\parindent}{0em}

%%%%%%%%%%%Configurations for code%%%%%%%%%%%%%%%%%%%%%%%
\SetKwInOut{Input}{Input}
\SetKwInOut{Output}{Output}
\SetKwProg{Fn}{Function}{\string:}{end}
\SetKwFunction{mstnew}{MST\_New}
\SetKwFunction{tw}{TreeWeight}
\SetKwFunction{dps}{DFS}
\SetKwFunction{con}{Is\_Connected}
\SetKwFunction{hor}{Three\_Fastest\_Horses}
%%%%%%%%%%%Here is the configurations for Code%%%%%%%%%%%

\definecolor{mygreen}{rgb}{0,0.6,0}
\definecolor{mygray}{rgb}{0.7,0.7,0.7}
\definecolor{mymauve}{rgb}{0.58,0,0.82}
\definecolor{mywhite}{rgb}{1,1,1}
\definecolor{myblack}{rgb}{0,0,0}
\definecolor{myblue}{RGB}{27,154,154}
\lstset{
backgroundcolor=\color{white},
basicstyle = \footnotesize\Inputmono,
breakatwhitespace = false,
breaklines = true,
captionpos = b,
commentstyle = \color{mygray}\bfseries,
extendedchars = false,
frame =shadowbox,
framerule=0.5pt,
frameround=tttt,
keepspaces=true,
keywordstyle=\color{myblue}\bfseries, % keyword style
language = Verilog,                     % the language of code
otherkeywords={string},
numbers=left,
numbersep=5pt,
numberstyle=\tiny\color{mymauve},
rulecolor=\color{black},
showspaces=false,
showstringspaces=false,
showtabs=false,
stepnumber=0,
stringstyle=\color{mymauve},        % string literal style
tabsize=2,
title=\lstname
}

%%%%%%%%%%%%%%%%%%%%%%%%%%%%%%%%%%%%%%%%%%%%

\begin{document}
%\setmainfont{Savoye LET}
%\setmainfont{Cormorant Upright}
\setmainfont{Cormorant Upright}
\renewcommand\arraystretch{1.5}


\thispagestyle{empty}

\begin{center}
\begin{large}
\begin{figure}[!htbp]
\centering
\includegraphics[width=0.7\textwidth]{Logo2}
\end{figure}
\hrule
\vspace*{0.25cm}
\sc{ \Large  UM--SJTU Joint Institute \vspace*{0.3em}} \\
\Large  VV557 Methods of Applied Math II\\
\end{large}
\hrulefill

\vspace*{2cm}
\begin{Large}
\sc{{Assignment 6}} \\
\end{Large}
\vspace*{2cm}
\begin{Large}
\sc{{Group 22}}\\
\end{Large}
\vspace*{0.5cm}
\begin{large}
\sc{{Sui, Zijian\ \ 515370910038}} \\
\sc{{Wang, Tianze\ \ 515370910202}} \\
\sc{{Xu, Yisu \ \ 118370910021}} \\
\end{large}
\end{center}
\newpage
\setmainfont{Optima}
\setmonofont{Optima}
\setsansfont{Optima}
%\tableofcontents
%\newpage
\setcounter{page}{1}
\normalsize
\section{}
Given that $v$ satisfies
\begin{equation}
	\int_{\partial \Omega} J(u,v) d \vec{\sigma} = \int_{\partial \Omega}p (u \frac{\partial v}{\partial n}- v \frac{\partial u}{dn}) = 0 \label{con:1.1}
\end{equation}
Since $u \in M$,
\begin{align*}
	u|_{\partial \Omega} = 0
\end{align*}
Plug in eqn. \eqref{con:1.1}, it yields to
\begin{equation}
	\int_{\partial \Omega} J(u,v) d \vec{\sigma} = \int_{\partial \Omega}p (0 \cdot  \frac{\partial v}{\partial n}- v \frac{\partial u}{\partial n}) d \vec{\sigma}= - \int_{\partial \Omega}p  v \frac{\partial u}{\partial n} d \vec{\sigma}\label{con:1.2} 
\end{equation}
From condition of $u$, we know nothing about the term $\frac{\partial u}{\partial n}$, which means it could be arbitrary. Also, by definition, $p>0$. To let equation \eqref{con:1.2} evaluate to 0, it must satisfy $v|_{\partial \Omega}=0$, which means
\[
 	v \in M
 \] 
 \qedhere
\section{}
\subsection{}
(a)
$$L = \frac { \partial ^ { 2 } } { \partial t ^ { 2 } } - \frac { \partial ^ { 2 } } { \partial x ^ { 2 } },
\quad  L ^ { * } = \frac { \partial ^ { 2 } } { \partial t ^ { 2 } } - \frac { \partial ^ { 2 } } { \partial x ^ { 2 } }=L
$$
$$\int _ { \Omega } \left( v L u - u L ^ { * } v \right) d ( x , t ) = \int _ { \Omega } ( v L u - u L v ) d ( x , t )= \int _ { \partial \Omega } J ( u , v ) d \sigma $$
Then it could be expressed as 
\begin{align*}
\int_{\partial \Omega} J(u, v) d \sigma &=\int_{\partial \Omega} \left( \begin{array}{c}{v \frac{\partial u}{\partial t}-u \frac{\partial v}{\partial t}} \\ {u \frac{\partial v}{\partial x}-v \frac{\partial u}{\partial x}}\end{array}\right) d \sigma \\
 &=\int_{0}^{T}\left.\left(u v_{x}-v u_{x}\right)\right|_{-L} ^{L} d t+\int_{-L}^{L}\left.\left(v u_{t}-u v_{t}\right)\right|_{0} ^{T} d x 
 \\ &=\int_{0}^{T} u\left.v_{x}\right|_{-L} ^{L} d t+\int_{-L}^{L}\left[v(x, T) u_{t}(x, T)-u(x, T) v_{t}(x, T)\right] d x
\end{align*}
Then the adjoint boundary conditions are
$$\left. \frac { \partial v } { \partial n } \right| _ { \partial I } = 0, v ( x , T ) = 0 , v _ { t } ( x , T ) = 0 $$

(b)
$$
J ( u , v ) =  \left( \begin{array} { l } { v \frac { \partial u } { \partial t } - u \frac { \partial v } {\partial t } }\\{ u \frac { \partial v } { \partial x } - v \frac { \partial u } { \partial x } }  \end{array} \right)
$$

(c)
The Green function satisfies
$$
L g  ( x , t ; \xi , \tau ) = \delta ( x - \xi )\delta ( t - \tau )
$$
Since $ Lu=F, and \left. \frac { \partial v } { \partial n } \right| _ { \partial I } = 0, u ( x , 0 ) = f ( x ) , \quad u _ { t } ( x , 0 ) = h ( x )$
We have,
$$
u ( \xi , \tau ) = \int _ { \Omega } g F d ( x , t )-\int _ { \partial \Omega } J \left( u , g \right) d \xi d\tau
$$
$$
u ( x , t ) = \int _ { \Omega } g(\xi,\tau;x,t) F(\xi, \tau) d ( \xi , \tau ) + \int _ { - L } ^ { L } \left[  h ( \xi ) g (\xi, 0; x,t ) + f ( \xi ) g _ { t } (\xi, 0; x,t ) \right] d \xi
$$

\subsection{}
$$ E ( x , t ; \xi , \tau ) = \frac { 1 } { 2 } H ( t - \tau - | x - \xi | )
$$

When\quad $ t=0, \tau > 0 , | x - \xi | \geq 0$ 
$$
E ( x , 0 ; \xi , \tau ) = \frac { 1 } { 2 } H ( - \tau - | x - \xi | )=0
$$
Since $$
T _ { \frac { \partial E } { \partial t } } \varphi = - \int _ { R } \frac { 1 } { 2 } H ( t - \tau - | x - \xi | ) \frac { \varphi ( t ) } { \partial t } d t= T _ { \frac { 1 } { 2 } \delta ( t - \tau - | x - \xi | ) } \varphi
$$,
$$E _ { t } ( x , t ; \xi , \tau ) = \delta ( t - \tau - | x - \xi | )$$

If L is Large enough,$$
 \left. \frac { \partial E } { \partial x } \right| _ { x = - L } =  \left. - \frac { 1 } { 2 } \operatorname { sgn } ( x - \xi ) \cdot \delta ( t - \tau - | x - \xi | ) \right| _ { x = -L } = -\frac { 1 } { 2 } \delta \left( t - \tau - L -\xi \right) = 0
$$
$$
\left. \frac { \partial E } { \partial x } \right| _ { x = L } =  \left. - \frac { 1 } { 2 } \operatorname { sgn } ( x - \xi ) \cdot \delta ( t - \tau - | x - \xi | ) \right| _ { x = L } = \frac { 1 } { 2 }  \delta \left( t - \tau - L +\xi \right) = 0
$$

$$
\left. \frac { \partial E } { \partial x } \right| _ { x = -L } = \left. \frac { \partial E } { \partial x } \right| _ { x = L } = 0, \quad \left. \frac { \partial E } { \partial n } \right| _ { \partial I } = 0$$
Therefore, if L is large enough, the fundamental solution$ E ( x , t ; \xi , \tau ) = \frac { 1 } { 2 } H ( t - \tau - | x - \xi | )$ satisfies the boundary conditions.
\subsection{}

Since E is the Green function and T is fixed, we have


\[
u\left(x,t\right)=\int_{\Omega{}}g(x,t;\xi{},\tau{})F(\xi{},\tau{})d\xi{}d\tau{}+\int_{-L}^Lh\left(\xi{}\right)g\left(x,t;\xi{},0\right)-f(\xi{})g_t\left(x,t;\xi{},0\right)d\xi{}
\]


From the results in i) and ii), and the additional boundary conditions (*), we
have


\[
\int_{-L}^Lh\left(\xi{}\right)g\left(x,t;\xi{},0\right)-f\left(\xi{}\right)g_t\left(x,t;\xi{},0\right)d\xi{}
\]



\[
=\int_{t=T}\left(-\frac{1}{2}H\left(t-\tau{}-\left\vert{}x-\xi{}\right\vert{}\right)\right)h\left(\xi{},\tau{}\right)+\frac{1}{2}f\left(\xi{},\tau{}\right)\delta{}\left(t-\tau{}-\left\vert{}x-\xi{}\right\vert{}\right)d\xi{}
\]



\[
=\frac{f\left(T-\tau{}+x,\tau{}\right)+f\left(x-T+\tau{},\tau{}\right)}{2}-\int_{x-t+\tau{}}^{t-\tau{}+x}h\left(\sigma{},\tau{}\right)d\sigma{}
\]



\[
=\frac{f\left(x+T\right)+f\left(x-T\right)}{2}+\frac{1}{2}\int_{x-t}^{x+t}h\left(\sigma{}\right)d\sigma{}
\]


While


\[
\int_{\Omega{}}g(x,t;\xi{},\tau{})F(\xi{},\tau{})d\xi{}d\tau{}=\int_{\Omega{}}\frac{1}{2}H(t-\vert{}x-\xi{}\vert{})F(\xi{},\tau{})d\xi{}d\tau{}=\frac{1}{2}\iint_{\Delta{}(x,t)}F(\xi{},\tau{})d\xi{}d\tau{}
\]


where
$\Delta{}\left(x,t\right)=\{(\xi{},\tau{})\in{}R^2:0\leq{}\tau{}\leq{}t-\vert{}x-\xi{}\vert{}\}$

When $\tau{}$=0,


\[
u\left(x,t\right)=\frac{f\left(x+t\right)+f\left(x-t\right)}{2}+\frac{1}{2}\int_{x-t}^{x+t}h\left(y\right)dy+\frac{1}{2}\iint_{\Delta{}(x,t)}F(y,s)dyds
\]


where
$\Delta{}\left(x,t\right)=\{(y,s)\in{}R^2:0\leq{}s\leq{}t-\vert{}x-y\vert{}\}$.

\end{document}
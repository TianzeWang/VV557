% !TEX program = xelatex
\documentclass{article}
\usepackage{geometry}
\geometry{left = 2.5cm, right = 2.5cm, top = 3cm, bottom = 3cm}
\usepackage[linesnumbered,ruled,longend]{algorithm2e}
\usepackage{amsmath}
\usepackage[makeroom]{cancel}
\usepackage{amsfonts,amssymb}
\usepackage{blkarray}
\usepackage{booktabs}
\usepackage{dsfont}
\usepackage{enumerate}
\usepackage{extarrows}
\usepackage{epsf}
\usepackage{fontspec}
\usepackage{forest}
\usepackage[colorlinks=true,linkcolor=purple]{hyperref}
\usepackage{listings}
\usepackage{mathrsfs}
\usepackage{microtype}
\usepackage{multirow}
\usepackage{setspace}
\usepackage{tikz}
%\usepackage{indentfirst}
%\usepackage[usenames,dvipsnames]{xcolor}
\newfontfamily\Inputmono{Consolas}
\renewcommand\thesection{Exercise 3.\ \arabic{section}}%\arabic{section}}
\renewcommand\thesubsection{\roman{subsection}).}
\renewcommand\thesubsubsection{\roman{subsection}).\alph{subsubsection}.}
\newcommand{\qedhere}{$\hfill\ensuremath{\square}$}
\newcommand{\f}{\frac}
\newcommand\supp{{\rm supp\ }}
\defaultfontfeatures{Mapping=tex-text,Scale=MatchLowercase}
\newcommand\mycommfont[1]{\ttfamily\textcolor{blue}{#1}}
\SetCommentSty{mycommfont}
%\setmainfont{Citadel Script}
%\setmainfont{Chalkboard}
\setmainfont{CMU Bright}
%\setmainfont{Apple Chancery}
\setmonofont{Optima}
\setsansfont{Optima}
%\renewcommand{\familydefault}{\sfdefault}
%\renewcommand{\footnotesize}{\sfdefault}
\setlength{\parskip}{0.25em}
\setlength{\parindent}{0em}

%%%%%%%%%%%Configurations for code%%%%%%%%%%%%%%%%%%%%%%%
\SetKwInOut{Input}{Input}
\SetKwInOut{Output}{Output}
\SetKwProg{Fn}{Function}{\string:}{end}
\SetKwFunction{mstnew}{MST\_New}
\SetKwFunction{tw}{TreeWeight}
\SetKwFunction{dps}{DFS}
\SetKwFunction{con}{Is\_Connected}
\SetKwFunction{hor}{Three\_Fastest\_Horses}
%%%%%%%%%%%Here is the configurations for Code%%%%%%%%%%%

%\definecolor{mygreen}{rgb}{0,0.6,0}
%\definecolor{mygray}{rgb}{0.7,0.7,0.7}
%\definecolor{mymauve}{rgb}{0.58,0,0.82}
%\definecolor{mywhite}{rgb}{1,1,1}
%\definecolor{myblack}{rgb}{0,0,0}
%\definecolor{myblue}{RGB}{27,154,154}
%\lstset{
% backgroundcolor=\color{white},
% basicstyle = \footnotesize\Inputmono,
% breakatwhitespace = false,
% breaklines = true,
% captionpos = b,
% commentstyle = \color{mygray}\bfseries,
% extendedchars = false,
% frame =shadowbox,
% framerule=0.5pt,
% frameround=tttt,
% keepspaces=true,
% keywordstyle=\color{myblue}\bfseries, % keyword style
% language = Verilog,                     % the language of code
% otherkeywords={string},
% numbers=left,
% numbersep=5pt,
% numberstyle=\tiny\color{mymauve},
% rulecolor=\color{black},
% showspaces=false,
% showstringspaces=false,
% showtabs=false,
% stepnumber=0,
% stringstyle=\color{mymauve},        % string literal style
% tabsize=2,
% title=\lstname
%}

%%%%%%%%%%%%%%%%%%%%%%%%%%%%%%%%%%%%%%%%%%%%

\begin{document}
%\setmainfont{Savoye LET}
%\setmainfont{Cormorant Upright}
\setmainfont{Cormorant Upright}
\renewcommand\arraystretch{1.5}


\thispagestyle{empty}

\begin{center}
\begin{large}
\begin{figure}[!htbp]
\centering
\includegraphics[width=0.7\textwidth]{Logo2}
\end{figure}
\hrule
\vspace*{0.25cm}
\sc{ \Large  UM--SJTU Joint Institute \vspace*{0.3em}} \\
\Large  VV557 Methods of Applied Math II\\
\end{large}
\hrulefill

\vspace*{2cm}
\begin{Large}
\sc{{Assignment 3}} \\
\end{Large}
\vspace*{2cm}
\begin{Large}
\sc{{Group 22}}\\
\end{Large}
\vspace*{0.5cm}
\begin{large}
\sc{{Sui, Zijian\ \ 515370910038}} \\
\sc{{Wang, Tianze\ \ 515370910202}} \\
\sc{{Xu, Yisu \ \ 118370910021}} \\
\end{large}
\end{center}
\newpage
\setmainfont{Optima}
\setmonofont{Optima}
\setsansfont{Optima}
%\tableofcontents
%\newpage
\setcounter{page}{1}
\normalsize
\section{Fourier Transform}
The Fourier Transform is defined as 
\[
	\mathcal{F}(\omega)=\frac{1}{\sqrt{2\pi}} \int_{-\infty}^{\infty} f(t) e^{-i \omega t} d t
\]
\subsection{}
Plug in the definition of $f(x)$
\[
	f(x)= \Pi_{a, b}(x)=\left\{\begin{array}{ll}{1} & {a<x<b} \\ {0} & {\text { otherwise }}\end{array}\right., \quad a, b \in \mathbb{R}
\]
The Fourier transform is then calculated as 
\begin{align*}
	\mathcal{F}(\omega)\cdot \sqrt{2\pi}&=\int_{-\infty}^{\infty} f(t) e^{-i \omega t} d t \\ 
	&=\int_{-\infty}^{a} 0\cdot  e^{-i \omega t} d t +\int_{a}^{b} e^{-i \omega t} d t +\int_{b}^{\infty} 0\cdot  e^{-i \omega t} d t \\
	&= \int_{a}^{b} e^{-i \omega t} d t  \\
	&= \left. \frac{e^{-i \omega t}}{-i \omega} \right|_a^b = \frac{e^{-i \omega b}-e^{-i \omega a}}{-i \omega}
\end{align*}
So
\[
	\mathcal{F}(\omega) = \frac{e^{-i \omega b}-e^{-i \omega a}}{-i \omega \sqrt{2\pi}}
\]
\subsection{}
\[
	f(x) = e^{-a|x|}
\]
Plug it in Fourier transform, which yields to 
\begin{align*}
	\mathcal{F}(\omega)\cdot \sqrt{2\pi}&=\int_{-\infty}^{\infty} f(t) e^{-i \omega t} d t \\ 
	&= \int_{-\infty}^0 e^{ax}  e^{-i \omega x} dx + \int_0^{+\infty} e^{-ax}  e^{-i \omega x} dx \\
	&= \left.\frac{e^{x(a- i \omega)}}{a- i \omega} \right|_{-\infty}^0 + \left.\frac{e^{-x (a+ i \omega)}}{-a - i \omega} \right|_0^{+\infty} \\
	& = -\frac{1}{a- i \omega}+ \frac{1}{-a- i \omega}\\
	&= \frac{1}{a- i \omega}+ \frac{1}{a+ i \omega} \\
	&= \frac{2a}{a^2+\omega^2}
\end{align*}
So 
\[
	\mathcal{F}(\omega)=\frac{a \sqrt{\frac{2}{\pi}}}{a^{2}+\omega^{2}}
\]
\subsection{}
Plug in $f(t) = e^{-a x^{2}}$
\begin{align*}
	\mathcal{F}(\omega)&=\int_{-\infty}^{\infty} f(t) e^{-i \omega t} d t \\ 
	&= \int_{-\infty}^{\infty} e^{-a t^{2}} e^{-i \omega t} d t \\ 
	&= \int_{-\infty}^{\infty} \mathrm{e}^{\frac{b^{2}}{4 a}-\left(\sqrt{a} t+\frac{b}{2 \sqrt{a}}\right)^{2}} \mathrm{d} t
\end{align*}
Then we apply the substitution Rule, we define $u:= (\sqrt{a} t+\frac{b}{2 \sqrt{a}})$, then 
\[
	\mathrm{d} u = \sqrt{a}\mathrm{d}t
\]
The original equation then becomes
\begin{align*}
	&\int_{-\infty}^{\infty} \mathrm{e}^{\frac{b^{2}}{4 a}-\left(\sqrt{a} t+\frac{b}{2 \sqrt{a}}\right)^{2}} \mathrm{d} t \\
	=&\int_{-\infty}^{\infty} \frac{\mathrm{e}^{\frac{b^{2}}{4 a}-u^{2}} }{\sqrt{a}}\mathrm{d}u\\
	=&\frac{\mathrm{e}^{\frac{b^{2}}{4 a}}}{\sqrt{a}}\int_{-\infty}^{\infty} \frac{\mathrm{e}^{-u^{2}}}{\sqrt{a}} du
\end{align*} 
Note that for the third step, we define
\[
	b = i \omega
\]
We recall the definition of Gauss Error function, which is of the similar form
\[
	\mathrm{erf}(x) = \frac{1}{\sqrt{\pi}} \int_{-x}^{x} e^{-t^{2}} d t
\]
which yields to
\begin{align*}
	\mathcal{F}( \omega)\cdot \sqrt{2\pi} =&\frac{\sqrt{\pi} \mathrm{e}^{\frac{b^{2}}{4 a}}}{ \sqrt{a}} \int_{-\infty}^{\infty} \frac{ \mathrm{e}^{-u^{2}}}{\sqrt{\pi}} \mathrm{d} u\\
	=& \frac{\sqrt{\pi} \mathrm{e}^{\frac{b^{2}}{4 a}}}{ \sqrt{a}} \mathrm{erf}(x)|_{0}^{+\infty}\\
	=&\frac{\sqrt{\pi} \mathrm{e}^{\frac{-\omega^2}{4 a}}}{ \sqrt{a}} (1-0)\\
	=&\frac{\sqrt{\pi} \mathrm{e}^{\frac{-\omega^2}{4 a}}}{\sqrt{a}}
\end{align*}
So 
\[
	\mathcal{F}( \omega)=\frac{e^{-\frac{w^{2}}{4 a}}}{\sqrt{2a}}
\]
\subsection{}
Plug in $f(x) = \cos (x) e^{-x^{2}}$,
\begin{align*}
	\mathcal{F}(\omega)\cdot \sqrt{2\pi} &=\int_{-\infty}^{\infty} f(t) e^{-i \omega t} d t \\ 	
	&=\int_{-\infty}^{\infty} \cos (t) e^{-t^{2}} e^{-i \omega t} d t \\
	&=\int_{-\infty}^{\infty} \frac{e^{it}+e^{-it}}{2} e^{-t^{2}} e^{-i \omega t} d t \\
	&=\int_{-\infty}^{\infty} \frac{e^{it}}{2} e^{-t^{2}} e^{-i \omega t} d t+\int_{-\infty}^{\infty} \frac{e^{-it}}{2} e^{-t^{2}} e^{-i \omega t} d t
\end{align*}
Then the strategy is mostly alike 3.1.iii). We simplify the procedure, and the answer is given as
\begin{align*}
	\mathcal{F}(\omega)\cdot \sqrt{2\pi} &=\frac{\sqrt{\pi} \mathrm{e}^{-\frac{w^{2}}{4}+\frac{w}{2}-\frac{1}{4}}}{2}+\frac{\sqrt{\pi} \mathrm{e}^{-\frac{u^{2}}{4}-\frac{w}{2}-\frac{1}{4}}}{2}\\
	&= \frac{\sqrt{\pi}\left(\mathrm{e}^{w}+1\right) \mathrm{e}^{-\frac{w^{2}}{4}-\frac{w}{2}-\frac{1}{4}}}{2}
\end{align*}
So
\[
	\mathcal{F}(\omega) = \frac{\left(\mathrm{e}^{w}+1\right) \mathrm{e}^{-\frac{w^{2}}{4}-\frac{w}{2}-\frac{1}{4}}}{2\sqrt{2}}
\]
\subsection{}
We consider the property that 
Plug in $f(x) = \cos (2 x) /\left(4+x^{2}\right)$
\begin{align*}
	\mathcal{F}(\omega)&=\frac{1}{\sqrt{2\pi}} \int_{-\infty}^{\infty} f(t) e^{-i \omega t} d t \\ 	
	&= \frac{1}{\sqrt{2\pi}} \int_{-\infty}^{\infty} \frac{\cos(2t)}{4+t^2} e^{-i \omega t} d t \\ 	
	&= \frac{1}{2\sqrt{2\pi}} \int_{-\infty}^{\infty} \frac{{e^{2it}+e^{-2it}}}{4+t^2} e^{-i \omega t} d t \\
	&= \frac{1}{2\sqrt{2\pi}} \left(\int_{-\infty}^{\infty} \frac{e^{2it}}{4+t^2} e^{-i \omega t} d t+\int_{-\infty}^{\infty} \frac{{e^{-2it}}}{4+t^2} e^{-i \omega t} d t\right)
\end{align*}
From the solution of 3.1.3, 
\[
	\widehat{e^{-a|x|}}=\frac{a \sqrt{\frac{2}{\pi}}}{a^{2}+\omega^{2}}
\]
Plug in $a = 2$, it yields to
\[
	\widehat{e^{-2|x|}}=\frac{2 \sqrt{\frac{2}{\pi}}}{4+\omega^{2}}
\]
From the definition of Fourier Transform
\[
	\widehat{\widehat{f(t)}}=f(-t)
\]
We could conclude that
\[
	\frac{1}{2\sqrt{2\pi}} \left(\int_{-\infty}^{\infty} \frac{e^{2it}}{4+t^2} e^{-i \omega t} d t\right) = \frac{1}{4}\sqrt{\frac{\pi}{2}} e^{-2|\omega-2|}
\]
So the whole Fourier Transform then becomes
\[
	\mathcal{F}(\omega) = \frac{1}{4}\sqrt{\frac{\pi}{2}} e^{-2|\omega-2|}+\frac{1}{4}\sqrt{\frac{\pi}{2}} e^{-2|\omega+2|}
\]
\subsection{}
Consider the property
\[
	\widehat{\varphi * \psi}=(2 \pi)^{n / 2} \hat{\varphi} \cdot \hat{\psi}
\]
Here $n=1$. Then
\[
	\widehat{x e^{-x^{2}}*e^{-x^{2}}} = (2 \pi)^{\frac{1}{2}} \widehat{x e^{-x^{2}}}\cdot \widehat{e^{-x^{2}}}
\]
From the result before
\[
	\widehat{e^{-x^{2}}}=\frac{e^{-\frac{w^{2}}{4}}}{\sqrt{2}}
\]
Then we apply Fourier Transform on $x e^{-x^{2}}$
\[
	\frac{1}{\sqrt{2\pi}}\int_{-\infty}^{\infty} t e^{-t^{2}}  e^{-i \omega t} d t = \frac{ \mathrm{i} \omega \mathrm{e}^{-\frac{w^{2}}{4}}}{2\sqrt{2}}
\]
So the total integral is given by
\begin{align*}
	\widehat{x e^{-x^{2}}*e^{-x^{2}}} &= (2 \pi)^{\frac{1}{2}}\cdot \frac{e^{-\frac{w^{2}}{4}}}{\sqrt{2}}\cdot  \frac{ \mathrm{i} \omega \mathrm{e}^{-\frac{w^{2}}{4}}}{2\sqrt{2}}\\
	&=\frac{\sqrt{2 \pi} e^{-\frac{w^{2}}{4}} i \omega e^{-\frac{w^{2}}{4}} }{4}
\end{align*}
\section{}
\subsection{}
$g(x) = \left\{\begin{array}{ll}{e^{-\varepsilon x}} & {x \geq 1} \\ {0} & {x<1}\end{array}\right. \quad \varepsilon>0$
\begin{align*}
	\hat{T}_{g} \varphi&=T_{g} \hat{\varphi}=\int_{1}^{\infty} e^{-\varepsilon \xi} \cdot \hat{\varphi}(\xi) d \xi\\
	&=\frac{1}{\sqrt{2 \pi}} \int_{1}^{\infty} \int_{-\infty}^{\infty} e^{-\varepsilon \xi} e^{-i x \xi} \varphi(x) d x d \xi \\
	&=\frac{1}{\sqrt{2 \pi}} \int_{1}^{\infty} \int_{-\infty}^{\infty} e^{(-\varepsilon -i x )\xi} \varphi(x) d x d \xi \\
	&= \frac{1}{\sqrt{2 \pi}}  \int_{-\infty}^{\infty} \left(\int_{1}^{\infty} e^{(-\varepsilon -i x )\xi} d \xi\right) \varphi(x) d x  \\
	&= \frac{1}{\sqrt{2 \pi}}  \int_{-\infty}^{\infty} \frac{e^{-\varepsilon-ix}}{\varepsilon+ix} \varphi(x) d x  \\
\end{align*}
So 
\[
	\hat{g(x)} =  \frac{1}{\sqrt{2 \pi}}\cdot \frac{e^{-\varepsilon-ix}}{\varepsilon+ix}
\]
\subsection{}
$$\sin ( 3 x - 2 ) = \frac { e ^ { i ( 3x - 2 ) } - e ^ { - i ( 3 x - 2 ) } } { 2 i } = \frac { e ^ { - 2 i } } { 2 i } e ^ { i 3 x } - \frac { e ^ { 2 i } } { 2 i } e ^ { - i 3 x }$$
	
$$\hat { T } _ { \delta \left( \xi - \xi _ { 0 } \right) } \varphi= \int _ { - \infty } ^ { \infty } \delta \left( \xi - \xi _ { 0 } \right) \cdot \int _ { - \infty } ^ { \infty } \frac { 1 } { \sqrt { 2 \pi } } e ^ { - i x \xi } \varphi ( x ) d x d\xi	= \int _ { - \infty } ^ { \infty } \frac { 1 } { \sqrt { 2 \pi } } e ^ { - i x \xi _ { 0 }} \varphi ( x ) d x$$
	
$$\hat { \delta } \left( \xi - \xi _ { 0 } \right) = \frac { 1 } { \sqrt { 2 \pi } } e ^ { - i \xi _ { 0 } x }$$,thus $$e ^ { - i \xi _ { 0 } x } = \sqrt { 2 \pi } \hat { \delta } \left( \xi - \xi _ { 0 } \right)$$
\begin{align*}
	\hat { T }_{\sin ( 3 x - 2 )}  \varphi &= T \frac { e ^ { - 2i } } { 2 i } \delta ( \xi + 3 ) \hat{\hat { \varphi }} - T \frac { e ^ { 2 i } } { 2 i } \delta ( \xi - 3 )\hat{\hat { \varphi }}	\\
&=\frac { e ^ { - 2 i } } { 2 i } \int _ { - \infty } ^ { \infty } \sqrt { 2 \pi } \delta ( \xi + 3 ) \varphi ( - \xi ) d \xi - \frac { e ^ { 2 i } } { 2 i } \int _ { - \infty } ^ { \infty } \sqrt { 2 \pi } \delta ( \xi - 3 ) \varphi ( - \xi ) d \xi\\
&= \sqrt { 2 \pi }\left[ \frac { e ^ { - 2 i} } { 2 i }\varphi ( 3 ) - \frac { e ^ { 2 i } } { 2 i } \varphi ( - 3 ) \right] 
\end{align*}	
which gives the final answer
$$\widehat { \sin ( 3x - 2 ) } =i  \sqrt { \frac { \pi } { 2 } } \left[ e ^ { 2 i  } \delta ( \xi + 3 ) - e ^ { - 2 i  } \delta ( \xi - 3 ) \right]$$

\subsection{}
$$\cos x = \frac { e ^ { i x } + e ^ { - i x } } { 2 }$$
So

\begin{align*}
\hat { T } _ { g } \varphi& = T _ { g } \hat { \varphi } = \int _ { - \infty } ^ { \infty } \xi ^ { 2 } \cos ( \xi ) \hat {\varphi }  ( \xi ) d \xi\\
&= \frac { 1 }{ \sqrt { 2 \pi } } \int _ { - \infty } ^ { \infty } \int _ { - \infty } ^ { \infty } \xi ^ { 2 } \cos ( \xi ) e ^ { - i x \xi } \varphi ( x ) d x d \xi\\
&=\frac { 1 } { \sqrt { 2 \pi } } \int _ { - \infty } ^ { \infty } \cos ( \xi ) \left[ \int _ { - \infty } ^ { \infty } \xi ^ { 2 } e ^ { - i \xi x } \varphi ( x ) d x \right] d \xi\\
&= \frac { 1 } { \sqrt { 2 \pi } } \int _ { - \infty } ^ { \infty } \frac { e ^ { i \xi } + e ^ { - i \xi } } { 2 } \left( \int _ { - \infty } ^ { \infty } - e ^ { - i \xi x } \varphi ^ { \prime \prime } ( x ) d x \right) d \xi\\
&=-\sqrt { \frac { \pi } { 2 } } \varphi ^ { \prime \prime } ( 1 ) - \sqrt { \frac { \pi } { 2 } } \varphi ^ { \prime \prime } ( - 1)
\end{align*}
which yields to the answer
$$ \widehat{x ^ { 2 } \cos x}= - \sqrt { \frac { \pi } { 2 } } \delta ^ { \prime \prime } ( \xi - 1 ) - \sqrt { \frac { \pi } { 2 } } \delta ^ { \prime \prime } ( \xi + 1 )$$

\end{document}

	















% !TEX program = xelatex
\documentclass{article}
\usepackage{geometry}
\geometry{left = 3cm, right = 3cm, top = 3cm, bottom = 3cm}
\usepackage[linesnumbered,ruled,longend]{algorithm2e}
\usepackage{amsmath}
\usepackage{amsfonts,amssymb}
\usepackage{blkarray}
\usepackage{booktabs}
\usepackage{dsfont}
\usepackage{enumerate}
\usepackage{epsf}
\usepackage{fontspec}
\usepackage{forest}
\usepackage[colorlinks=true,linkcolor=purple]{hyperref}
\usepackage{listings}
\usepackage{mathrsfs}
\usepackage{microtype}
\usepackage{multirow}
\usepackage{setspace}
\usepackage{tikz}
%\usepackage{indentfirst}
%\usepackage[usenames,dvipsnames]{xcolor}
\newfontfamily\Inputmono{Consolas}
\renewcommand\thesection{Exercise 1.\ \arabic{section}}%\arabic{section}}
\renewcommand\thesubsection{\roman{subsection}).}
\renewcommand\thesubsubsection{\arabic{subsubsection}.}
\newcommand{\qedhere}{$\hfill\ensuremath{\square}$}
\defaultfontfeatures{Mapping=tex-text,Scale=MatchLowercase}
\newcommand\mycommfont[1]{\ttfamily\textcolor{blue}{#1}}
\SetCommentSty{mycommfont}
%\setmainfont{Citadel Script}
%\setmainfont{Chalkboard}
\setmainfont{CMU Bright}
%\setmainfont{Apple Chancery}
\setmonofont{Optima}
\setsansfont{Optima}
%\renewcommand{\familydefault}{\sfdefault}
%\renewcommand{\footnotesize}{\sfdefault}
\setlength{\parskip}{0.25em}
\setlength{\parindent}{0em}

%%%%%%%%%%%Configurations for code%%%%%%%%%%%%%%%%%%%%%%%
\SetKwInOut{Input}{Input}
\SetKwInOut{Output}{Output}
\SetKwProg{Fn}{Function}{\string:}{end}
\SetKwFunction{mstnew}{MST\_New}
\SetKwFunction{tw}{TreeWeight}
\SetKwFunction{dps}{DFS}
\SetKwFunction{con}{Is\_Connected}
\SetKwFunction{hor}{Three\_Fastest\_Horses}
%%%%%%%%%%%Here is the configurations for Code%%%%%%%%%%%

%\definecolor{mygreen}{rgb}{0,0.6,0}
%\definecolor{mygray}{rgb}{0.7,0.7,0.7}
%\definecolor{mymauve}{rgb}{0.58,0,0.82}
%\definecolor{mywhite}{rgb}{1,1,1}
%\definecolor{myblack}{rgb}{0,0,0}
%\definecolor{myblue}{RGB}{27,154,154}
%\lstset{
% backgroundcolor=\color{white},
% basicstyle = \footnotesize\Inputmono,
% breakatwhitespace = false,
% breaklines = true,
% captionpos = b,
% commentstyle = \color{mygray}\bfseries,
% extendedchars = false,
% frame =shadowbox,
% framerule=0.5pt,
% frameround=tttt,
% keepspaces=true,
% keywordstyle=\color{myblue}\bfseries, % keyword style
% language = Verilog,                     % the language of code
% otherkeywords={string},
% numbers=left,
% numbersep=5pt,
% numberstyle=\tiny\color{mymauve},
% rulecolor=\color{black},
% showspaces=false,
% showstringspaces=false,
% showtabs=false,
% stepnumber=0,
% stringstyle=\color{mymauve},        % string literal style
% tabsize=2,
% title=\lstname
%}

%%%%%%%%%%%%%%%%%%%%%%%%%%%%%%%%%%%%%%%%%%%%

\begin{document}
%\setmainfont{Savoye LET}
%\setmainfont{Cormorant Upright}
\setmainfont{Cormorant Upright}
\renewcommand\arraystretch{1.5}


\thispagestyle{empty}

\begin{center}
\begin{large}
\begin{figure}[!htbp]
\centering
\includegraphics[width=0.7\textwidth]{Logo2}
\end{figure}
\hrule
\vspace*{0.25cm}
\sc{ \Large  UM--SJTU Joint Institute \vspace*{0.3em}} \\
\Large  VV557 Methods of Applied Math II\\
\end{large}
\hrulefill

\vspace*{2cm}
\begin{Large}
\sc{{Assignment 1}} \\
\end{Large}
\vspace*{2cm}
\begin{Large}
\sc{{Group 22}}\\
\end{Large}
\vspace*{0.5cm}
\begin{large}
\sc{{Xu, Yisu \ \ 118370910021}} \\
\sc{{Sui, Zijian\ \ 515370910038}} \\
\sc{{Wang, Tianze\ \ 515370910202}} \\
\end{large}
\end{center}
\newpage
\setmainfont{Optima}
\setmonofont{Optima}
\setsansfont{Optima}
%\tableofcontents
%\newpage
\setcounter{page}{1}
\section{}
\subsection{}
\normalsize
For the intervals on 
\[
	0 \leq x \leq \xi - \frac{1}{2n}\  \cup\ \xi + \frac{1}{2n} \leq x \leq 1
\]
the equation is given as 
\[
	- u'' = 0, \ \ u(0) = u(1) = 0
\]
And this is the same as the case on lecture slide pp. 24, obviously the solution is given as 
\[
	g(x,\xi) = \left\{
	\begin{aligned}
	&(1- \xi) \cdot  x \hspace*{1em} &0 \leq x \leq \xi - \frac{1}{2n} \\
	&\xi \cdot  (1-x) & \xi + \frac{1}{2n} \leq x \leq 1
	\end{aligned}
	\right.
\]
Next we need to discuss about the case about
\[
	f_n(x; \xi) = n \ \ \ \text{for} \ \ \xi - \frac{1}{2n} \leq x \leq \xi + \frac{1}{2n}
\]
\end{document}






% !TEX program = xelatex
\documentclass{article}
\usepackage{geometry}
\geometry{left = 3cm, right = 3cm, top = 3cm, bottom = 3cm}
\usepackage[linesnumbered,ruled,longend]{algorithm2e}
\usepackage{amsmath}
\usepackage{amsfonts,amssymb}
\usepackage{blkarray}
\usepackage{booktabs}
\usepackage{dsfont}
\usepackage{enumerate}
\usepackage{epsf}
\usepackage{fontspec}
\usepackage{forest}
\usepackage[colorlinks=true,linkcolor=purple]{hyperref}
\usepackage{listings}
\usepackage{mathrsfs}
\usepackage{microtype}
\usepackage{multirow}
\usepackage{setspace}
\usepackage{tikz}
%\usepackage{indentfirst}
%\usepackage[usenames,dvipsnames]{xcolor}
\newfontfamily\Inputmono{Consolas}
\renewcommand\thesection{Exercise 1.\ \arabic{section}}%\arabic{section}}
\renewcommand\thesubsection{\roman{subsection}).}
\renewcommand\thesubsubsection{\roman{subsection}).\alph{subsubsection}.}
\newcommand{\qedhere}{$\hfill\ensuremath{\square}$}
\defaultfontfeatures{Mapping=tex-text,Scale=MatchLowercase}
\newcommand\mycommfont[1]{\ttfamily\textcolor{blue}{#1}}
\SetCommentSty{mycommfont}
%\setmainfont{Citadel Script}
%\setmainfont{Chalkboard}
\setmainfont{CMU Bright}
%\setmainfont{Apple Chancery}
\setmonofont{Optima}
\setsansfont{Optima}
%\renewcommand{\familydefault}{\sfdefault}
%\renewcommand{\footnotesize}{\sfdefault}
\setlength{\parskip}{0.25em}
\setlength{\parindent}{0em}

%%%%%%%%%%%Configurations for code%%%%%%%%%%%%%%%%%%%%%%%
\SetKwInOut{Input}{Input}
\SetKwInOut{Output}{Output}
\SetKwProg{Fn}{Function}{\string:}{end}
\SetKwFunction{mstnew}{MST\_New}
\SetKwFunction{tw}{TreeWeight}
\SetKwFunction{dps}{DFS}
\SetKwFunction{con}{Is\_Connected}
\SetKwFunction{hor}{Three\_Fastest\_Horses}
%%%%%%%%%%%Here is the configurations for Code%%%%%%%%%%%

%\definecolor{mygreen}{rgb}{0,0.6,0}
%\definecolor{mygray}{rgb}{0.7,0.7,0.7}
%\definecolor{mymauve}{rgb}{0.58,0,0.82}
%\definecolor{mywhite}{rgb}{1,1,1}
%\definecolor{myblack}{rgb}{0,0,0}
%\definecolor{myblue}{RGB}{27,154,154}
%\lstset{
% backgroundcolor=\color{white},
% basicstyle = \footnotesize\Inputmono,
% breakatwhitespace = false,
% breaklines = true,
% captionpos = b,
% commentstyle = \color{mygray}\bfseries,
% extendedchars = false,
% frame =shadowbox,
% framerule=0.5pt,
% frameround=tttt,
% keepspaces=true,
% keywordstyle=\color{myblue}\bfseries, % keyword style
% language = Verilog,                     % the language of code
% otherkeywords={string},
% numbers=left,
% numbersep=5pt,
% numberstyle=\tiny\color{mymauve},
% rulecolor=\color{black},
% showspaces=false,
% showstringspaces=false,
% showtabs=false,
% stepnumber=0,
% stringstyle=\color{mymauve},        % string literal style
% tabsize=2,
% title=\lstname
%}

%%%%%%%%%%%%%%%%%%%%%%%%%%%%%%%%%%%%%%%%%%%%

\begin{document}
%\setmainfont{Savoye LET}
%\setmainfont{Cormorant Upright}
\setmainfont{Cormorant Upright}
\renewcommand\arraystretch{1.5}


\thispagestyle{empty}

\begin{center}
\begin{large}
\begin{figure}[!htbp]
\centering
\includegraphics[width=0.7\textwidth]{Logo2}
\end{figure}
\hrule
\vspace*{0.25cm}
\sc{ \Large  UM--SJTU Joint Institute \vspace*{0.3em}} \\
\Large  VV557 Methods of Applied Math II\\
\end{large}
\hrulefill

\vspace*{2cm}
\begin{Large}
\sc{{Assignment 1}} \\
\end{Large}
\vspace*{2cm}
\begin{Large}
\sc{{Group 22}}\\
\end{Large}
\vspace*{0.5cm}
\begin{large}
\sc{{Sui, Zijian\ \ 515370910038}} \\
\sc{{Wang, Tianze\ \ 515370910202}} \\
\sc{{Xu, Yisu \ \ 118370910021}} \\
\end{large}
\end{center}
\newpage
\setmainfont{Optima}
\setmonofont{Optima}
\setsansfont{Optima}
%\tableofcontents
%\newpage
\setcounter{page}{1}
\section{}
\subsection{}
\normalsize
% For the intervals on 
% \[
% 	0 \leq x \leq \xi - \frac{1}{2n}\  \cup\ \xi + \frac{1}{2n} \leq x \leq 1
% \]
% the equation is given as 
% \[
% 	- u'' = 0, \ \ u(0) = u(1) = 0
% \]
% And this is the same as the case on lecture slide pp. 24, obviously the solution is given as 
% \[
% 	g(x,\xi) = \left\{
% 	\begin{aligned}
% 	&(1- \xi) \cdot  x \hspace*{1em} &0 \leq x \leq \xi - \frac{1}{2n} \\
% 	&\xi \cdot  (1-x) & \xi + \frac{1}{2n} \leq x \leq 1
% 	\end{aligned}
% 	\right.
% \]
% Next we need to discuss about the case about
% \[
% 	f_n(x; \xi) = n \ \ \ \text{for} \ \ \xi - \frac{1}{2n} \leq x \leq \xi + \frac{1}{2n}
% \]
% Thus $u_n$ will be in the form of 
% \[
% 	u_n'' = -n \ \Rightarrow \ u_n = -\frac{n\cdot x^2}{2}+ a\cdot x+ b
% \]
% Since $u_n$ should satisfy the continuity condition for its original form and first derivative, 
% \[
	
% \]
For the intervals on 
\[
	0 \leq x \leq \xi - \frac{1}{2n}\  \cup\ \xi + \frac{1}{2n} \leq x \leq 1
\]
We have 
\[
	u_n(x)'' = 0 \Rightarrow \left\{\begin{aligned}
	&u_{n1}(x) = x^2+ a\cdot x+ b\ \  & 0 \leq x \leq \xi - \frac{1}{2n} \\\
 	&u_{n2}(x) = x^2+ c\cdot x+ d & \xi + \frac{1}{2n} \leq x \leq 1
	\end{aligned}\right.
\]
For the intervals on 
\[
	\xi - \frac{1}{2n} \leq x \leq \xi + \frac{1}{2n}
\]
The function is given as
\[
	u_{n3}(x)'' = -n \Rightarrow u_{n3}(x) = -\frac{n\cdot x^2}{2}+ \alpha\cdot x+ \beta
\]
For $u_{n1}, u_{n2}, u_{n3}$, they should satisfy
\[	\left\{
	\begin{aligned}
	&u_{n1}(\xi- \frac{1}{2n}) = u_{n3} (\xi - \frac{1}{2n})\\
	&u_{n2}(\xi+ \frac{1}{2n}) = u_{n3} (\xi + \frac{1}{2n})\\
	&\underset{x\rightarrow \xi- \frac{1}{2n}^-}{lim} u_{n1}'(x) =  \underset{x\rightarrow \xi- \frac{1}{2n}^+}{lim} u_{n3}'(x)\\
	&\underset{x\rightarrow \xi+ \frac{1}{2n}^-}{lim} u_{n3}'(x) =  \underset{x\rightarrow \xi+ \frac{1}{2n}^+}{lim} u_{n2}'(x)\\
	\end{aligned}
	\right.
\]
This will give a linear equation set (four equations) with four variables. ($a=0,d=0$)
And the final result will be then the solution provided:
\[
	u_n(x) = \left\{
	\begin{aligned}
	&(1- \xi) \cdot  x  & 0 \leq x \leq \xi - \frac{1}{2n} \\
	&(1- \xi) \cdot x - \frac{n}{2}(x- \xi+1/(2n))^2  & \xi - \frac{1}{2n} < x < \xi + \frac{1}{2n} \\
	&\xi \cdot  (1-x) & \xi + \frac{1}{2n} \leq x \leq 1
	\end{aligned}
	\right.
\]
\subsection{}
% According to the definition of pointwise convergence, we need to prove
% \[
% \begin{aligned}
% 	\underset{n\rightarrow \infty}{lim} |u_n(x) - g(x,\xi)| = 0
% \end{aligned}
% \]
As $n\rightarrow \infty$, 
\[
	\xi - \frac{1}{2n} = \xi =  \xi + \frac{1}{2n}
\]
And the term $$(1- \xi) \cdot x - \frac{n}{2}(x- \xi+1/(2n))^2  x  $$ is canceled as $n \rightarrow \infty$, then
\[
	\underset{n\rightarrow \infty}{lim} u_n(x) = \left\{\begin{aligned}
	&(1- \xi) x \ \ & 0 \leq x \leq \xi \\
	&\xi (1-x) & \xi \leq x \leq 1
\end{aligned}\right.
\]
Then since $\underset{n\rightarrow \infty}{lim} u_n(x) = g(x; \xi)$, it's sufficient to say
\[
\begin{aligned}
	\underset{n\rightarrow \infty}{lim} |u_n(x) - g(x,\xi)| = 0
\end{aligned}
\]
\qedhere

\subsection{}
No, it's not uniform convergence.
The definition of uniform convergence is 
\[
	\forall \varepsilon>0,\ \ \exists N_0\in N,\ \ \forall n > N_0,\ \  |f_n(x)-f(x)|<\varepsilon 
\]
Obviously, the convergence is uniform on $0 \leq x \leq \xi - \frac{1}{2n}$ and $\xi + \frac{1}{2n} \leq x \leq 1$. Since $u_n(x) = g(x;\xi)$ on these two intervals.\\ \\  
Next we will prove on $\xi - \frac{1}{2n} < x < \xi + \frac{1}{2n}$, it's not uniform convergence. We choose $\varepsilon = \varepsilon_0 < 1/4$, and we consider the interval on $[\xi, \xi + \frac{1}{2n} )$.
\[
	\begin{aligned}
	|u_n(x) - g(x;\xi)| &= |(1- \xi) \cdot x - \frac{n}{2}(x- \xi+1/(2n))^2 - (1-x)\xi | \\
	&= |t - \frac{n}{2}(t+\frac{1}{2n})^2| \ \ \ , t\leftarrow (x - \xi)
	\end{aligned}
\]
Since $\forall x$ on the interval,
\[
	|u_n(x) - g(x;\xi)| < \varepsilon_0
\]
However, we evaluate $x = \xi + \frac{1}{2n} - \varepsilon_1$
\[
	|u_n(x) - g(x;\xi)| = |\frac{1-n}{2n}| \rightarrow \frac{1}{2} > \varepsilon_0
\]
which means it is not uniform convergence.
\section{}
\subsection{}
We first test whether $T$ is a linear functional.
\[
	T(\lambda \varphi_1 + \mu \varphi_2) = \lambda \varphi_1(-10) + \lambda \varphi_2(-10) = \lambda T \varphi_1+ \mu T \varphi_2
\]
It's linear. Then we test its continuity
\[
	\varphi_m \rightarrow 0, \ |T \varphi_m| = |\varphi_m(-10)| \leq |\sup  \varphi_m(x)| = 0 
\]
So we can conclude \textbf{it's a distribution}.
\subsection{}
We first test whether $T$ is a linear functional.
\[
\begin{aligned}
	T(\lambda \varphi_1 + \mu \varphi_2) &= [\lambda \varphi_1(0) + \mu \varphi_2(0)]^2 \\
	& = \lambda^2 \varphi_1^2(0) + \mu^2 \varphi_2^2(0)+ 2 \lambda \mu \varphi_1(0) \varphi_2 (0) \\
	& \neq \lambda \varphi_1^2(0) + \mu \varphi_2^2(0) 
\end{aligned}
\]
So 
\[
	T(\lambda \varphi_1 + \mu \varphi_2) \neq \lambda T \varphi_1+ \mu T \varphi_2
\]
which means \textbf{it's not a distribution}.
\subsection{}
Obviously, $T$ maps to $\mathbb{C}^n$ instead of $\mathbb{C}$, so it's not a linear functional, thus \textbf{it's not a distribution}.
\newpage
\subsection{}
We first test whether $T$ is a linear functional.
\[
\begin{aligned}
	T(\lambda \varphi_1 + \mu \varphi_2) &=  \lambda \varphi_1(0)+ \mu \varphi_2(0) + \lambda \varphi_1(1)+ \mu \varphi_2(1) + \lambda \varphi_1(2)+ \mu \varphi_2(2) + ... \\
	& = \lambda (\varphi_1(0) + \varphi_1(1)+ \varphi_1(2) + ...) + \mu(\varphi_2(0) + \varphi_2(1)+ \varphi_2(2) + ...) \\
	& = \lambda T \varphi_1+ \mu T \varphi_2
\end{aligned}
\]
It's linear. Then we test its continuity
\begin{align*}
\varphi_m \rightarrow 0, \ \ |T \varphi_m| = |\phi_m(0)+ \phi_m(1)+ \phi_m(2)+ ...| \leq 0+0+0+...+0 = 0
\end{align*}
So we can conclude \textbf{it's a distribution}.
\subsection{}
We first test whether $T$ is a linear functional.
\begin{align*}
T(\lambda \varphi_1 + \mu \varphi_2) &= \int_{S^{n-1}} (\lambda \varphi_1 + \mu \varphi_2) \\
	&= \int_{S^{n-1}} \lambda \varphi_1 +\int_{S^{n-1}}  \mu \varphi_2 \\
	&= \lambda T \varphi_1+ \mu T \varphi_2
\end{align*}
It's linear. Then we test its continuity
\begin{align*}
\varphi_m \rightarrow 0, \ \ |T \varphi_m| = |\int_{S^{n-1}}  \varphi_m| \leq \sup|\varphi_m| = 0
\end{align*}
So we can conclude \textbf{it's a distribution}.
\subsection{}
\subsubsection{}
We first test whether $T$ is a linear functional.
\begin{align*}
T(\lambda \varphi_1 + \mu \varphi_2) &= \int \frac{\lambda \varphi_1 + \mu \varphi_2}{x} \\
	&= \int \frac{\lambda \varphi_1}{x} + \int \frac{\mu \varphi_2}{x} \\
	&= \lambda T \varphi_1+ \mu T \varphi_2
\end{align*}
It's linear. Then we test its continuity
\begin{align*}
\varphi_m \rightarrow 0, \ \ |T \varphi_m| & = |\int \frac{1}{x}  \varphi_m dx| \\
&= |\left.\ln(x) \varphi_m(x)\right| _{-\infty}^{+ \infty} - \int \ln|x| \varphi'(x) dx| \\
&= \int \ln|x| \varphi'(x) dx
\end{align*}
As $m \rightarrow \infty$, \[
	\varphi_m'(x) \rightarrow 0,\ \  \sup|D^\alpha| \varphi_m(x) \rightarrow 0 
\]	
So we could conclude it's a distribution.
\subsubsection{}
\begin{align*}
\varphi_m \rightarrow 0, \ \ |T \varphi_m| & = |\int \frac{1}{\sqrt{|x|}}  \varphi_m(x) dx| \\
& \leq (\int \frac{1}{\sqrt{|x|}}dx)\cdot sup|\varphi_m(x)| \\
& = 2 \sqrt{|x|} \cdot sup|\varphi_m(x)|
\end{align*}
So we can conclude it's a distribution.
\subsubsection{}
\begin{align*}
\varphi_m \rightarrow 0, \ \ |T \varphi_m| & = |\int \frac{1}{x^2}  \varphi_m(x) dx| \\
& = \left.- \frac{1}{x} \varphi(x) \right|_{-\infty}^\infty - \int (-\frac{1}{x}) \varphi'(x)dx \\
&= ln|x| \varphi'(x)|_{-\infty}^\infty - \int ln|x| \phi''(x) dx \\
&= \int ln|x| \varphi''(x) dx
\end{align*}
Sine \[
	sup|D^\alpha| \varphi_m(x) \rightarrow 0
\]
So we can conclude it's a distribution.

\end{document}























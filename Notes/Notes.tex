% !TEX program = xelatex
\documentclass{article}
\usepackage{geometry}
\geometry{left = 2.5cm, right = 2.5cm, top = 3cm, bottom = 3cm}
\usepackage[linesnumbered,ruled,longend]{algorithm2e}
\usepackage{amsmath}
\usepackage[makeroom]{cancel}
\usepackage{amsfonts,amssymb}
\usepackage{tcolorbox}
\usepackage{blkarray}
\usepackage{booktabs}
\usepackage{colortbl}
\usepackage{dsfont}
\usepackage{enumerate}
\usepackage{extarrows}
\usepackage{epsf}
\usepackage{fontspec}
\usepackage{forest}
\usepackage[colorlinks=true,linkcolor=purple]{hyperref}
\usepackage{listings}
\usepackage{mathrsfs}
\usepackage{microtype}
\usepackage{multirow}
\usepackage{setspace}
\usepackage{tikz}
%\usepackage{indentfirst}
%\usepackage[usenames,dvipsnames]{xcolor}
\newfontfamily\Inputmono{Consolas}
\renewcommand\thesection{Notes\ \arabic{section}}%\arabic{section}}
\renewcommand\thesubsection{\roman{subsection}).}
% \renewcommand\thesubsubsection{\roman{subsection}).\alph{subsubsection}.}
\newcommand{\qedhere}{$\hfill\ensuremath{\square}$}
\newcommand{\f}{\frac}
\newcommand\supp{{\rm supp\ }}
\defaultfontfeatures{Mapping=tex-text,Scale=MatchLowercase}
\newcommand\mycommfont[1]{\ttfamily\textcolor{blue}{#1}}
\SetCommentSty{mycommfont}
%\setmainfont{Citadel Script}
%\setmainfont{Chalkboard}
\setmainfont{CMU Bright}
%\setmainfont{Apple Chancery}
\setmonofont{Optima}
\setsansfont{Optima}
%\renewcommand{\familydefault}{\sfdefault}
%\renewcommand{\footnotesize}{\sfdefault}
\setlength{\parskip}{0.25em}
\setlength{\parindent}{0em}

%%%%%%%%%%%Configurations for code%%%%%%%%%%%%%%%%%%%%%%%
\SetKwInOut{Input}{Input}
\SetKwInOut{Output}{Output}
\SetKwProg{Fn}{Function}{\string:}{end}
\SetKwFunction{mstnew}{MST\_New}
\SetKwFunction{tw}{TreeWeight}
\SetKwFunction{dps}{DFS}
\SetKwFunction{con}{Is\_Connected}
\SetKwFunction{hor}{Three\_Fastest\_Horses}
%%%%%%%%%%%Here is the configurations for Code%%%%%%%%%%%

\definecolor{mygreen}{rgb}{0,0.6,0}
\definecolor{mygray}{rgb}{0.7,0.7,0.7}
\definecolor{mymauve}{rgb}{0.58,0,0.82}
\definecolor{mywhite}{rgb}{1,1,1}
\definecolor{myblack}{rgb}{0,0,0}
\definecolor{myblue}{RGB}{27,154,154}
\lstset{
backgroundcolor=\color{white},
basicstyle = \footnotesize\Inputmono,
breakatwhitespace = false,
breaklines = true,
captionpos = b,
commentstyle = \color{mygray}\bfseries,
extendedchars = false,
frame =shadowbox,
framerule=0.5pt,
frameround=tttt,
keepspaces=true,
keywordstyle=\color{myblue}\bfseries, % keyword style
language = Verilog,                     % the language of code
otherkeywords={string},
numbers=left,
numbersep=5pt,
numberstyle=\tiny\color{mymauve},
rulecolor=\color{black},
showspaces=false,
showstringspaces=false,
showtabs=false,
stepnumber=0,
stringstyle=\color{mymauve},        % string literal style
tabsize=2,
title=\lstname
}

%%%%%%%%%%%%%%%%%%%%%%%%%%%%%%%%%%%%%%%%%%%%

\begin{document}
%\setmainfont{Savoye LET}
%\setmainfont{Cormorant Upright}
\setmainfont{Cormorant Upright}
\renewcommand\arraystretch{1.5}


\thispagestyle{empty}

\begin{center}
\begin{large}
\begin{figure}[!htbp]
\centering
\includegraphics[width=0.7\textwidth]{Logo2}
\end{figure}
\hrule
\vspace*{0.25cm}
\sc{ \Large  UM--SJTU Joint Institute \vspace*{0.3em}} \\
\Large  VV557 Methods of Applied Math II\\
\end{large}
\hrulefill

\vspace*{2cm}
\begin{Large}
\sc{{Lecture Notes Hints}} \\
\end{Large}
\vspace*{2cm}
\begin{Large}

\end{Large}
\vspace*{0.5cm}
\begin{large}
\sc{collected by \textbf{Tianze, Wang}}
\end{large}
\end{center}
\newpage

\begin{center}
\vspace*{6em}
Page Left blank because I want to.
\end{center}

\newpage
\setmainfont{Optima}
\setmonofont{Optima}
\setsansfont{Optima}
%\tableofcontents
%\newpage
\setcounter{page}{1}
\normalsize
\section{PP 330}
To calculate a surface integral in $\mathbb{R}^n$ of $S\subset \mathbb{R}^n$ is a hypersurface.
\subsection{Step 1: Find parametrization $\gamma$}
\begin{align*}
	\gamma :[a_1,b_1]\times ... \times [a_{n+1},b_{n+1}] \rightarrow S \subset \mathbb{R}^n \\
	\gamma: (s_1,...,s_{n+1}) \mapsto \begin{pmatrix}
	\gamma_1(s_1,...s_{n+1})\\
	\gamma_2(s_1,...s_{n+1})\\
	\gamma_3(s_1,...s_{n+1})\\
	\vdots \\
	\gamma_{n+1}(s_1,...s_{n+1})\\
	\end{pmatrix} 
\end{align*}

\begin{tcolorbox}[colback=blue!20!white]
A simple example is given as
\begin{align*}
(\phi, \theta) = \begin{pmatrix}
	\cos \pi \sin \theta\\
	\sin \pi \sin \theta\\
	\cos \theta
\end{pmatrix}
\end{align*}
\end{tcolorbox}
\subsection{Step 2: Find a normal vector $\vec{n}\perp S$}
\subsection{}
\begin{align*}
	\int_S f ds = \int \int \int ...\int f(\gamma(s_1,...,s_{n+1})) \det(\frac{\partial \gamma}{\partial s_1},...,\vec{a})
\end{align*}

\section{PP 366}
\begin{align*}
	\int_{\varphi} f ds = \int_a^b f \circ \gamma(t)\cdot  |\gamma'(t)|dt
\end{align*}
where $\varphi$ indicates curve/line integral on $\mathbb{R}^n$. by
\[
	\gamma (t) = \begin{pmatrix}
		t \\ 0
	\end{pmatrix}
\]
$t \in R$, $\gamma'(t) = \begin{pmatrix} 1\\0 \end{pmatrix}$, $|\gamma'(t) | = 1$.
So
\begin{align*}
	&\int_{\partial \mathbb{H}} h(\cdot ) \frac{\partial g(x;\cdot )}{\partial n} ds \\
	&= \int_{-\infty}^\infty h(\gamma(t))\cdot \frac{-\partial g(x;\cdot )}{\partial \xi_2}|_{\gamma(t)} dt \\
	&= - \int _{- \infty }^{\infty}h(\xi_2)\cdot \frac{-\partial g(x;\xi_1,\xi_2 )}{\partial \xi_1}|_{\xi_2 = 0} dt \\
\end{align*}

\section{Method of Images}
\begin{tcolorbox}[colback=green!10!white]
This part is to raise an intuitive understanding of \textbf{Method of Images} chapter.
\end{tcolorbox}
Electrostatics by Maxwell
\[
	- \Delta \underbrace{V}_{\text{potential}} = 4 \pi \underbrace{\rho}_{\text{charge density}}
\]
For a unit point charge located at $\xi$, it has charge density $\delta(x- \xi)$. The potential for a point charge is
\begin{align*}
	V(x;\xi) = \frac{1}{4 \pi \varepsilon_0}\cdot  \frac{1}{|x-\xi|}
\end{align*}

This solves $-\Delta V = 4 \pi \delta(x-\xi)$

\begin{tcolorbox}[colback = blue!10!white]
E.g. a charge of 1 Coulomb at $\xi$ and 2 Coulomb at $\xi^*$,
\[
	- \Delta V = 4 \pi (\delta(x- \xi)+ 2 \delta(x- \xi^*))
\]
which is also taken as 
\[
	V = V_1 + V_2
\]
where according to the superposition principle
\begin{align}
	- \Delta V_1 = - 4 \pi \delta(x- \xi) \label{eqn1} \\
	- \Delta V_2 = - 4 \pi \delta(x- \xi^*) \label{eqn2} 
\end{align}
If $V_1$ solves equation \eqref{eqn1}, $V_2$ also solves equation \eqref{eqn2}. 
\[
	V_2(x; \xi^*) = 2 V_1(x;\xi^*)
\]
The effect of having a \textbf{ground plate} at x plane is equal to have a $\xi^*$ charge.
\end{tcolorbox}
\section{Neumann Problem}
\begin{align}
	\frac{\partial g}{\partial n} = 0 \quad g = \text{physical potential} \label{eqn4}
\end{align}
which means g satisfy $\Delta g = \delta$ and boundary condition.
\begin{equation}	
	\langle\nabla g, \vec{n}\rangle = 0 \label{eqn3}
\end{equation}
This equation means the force perpendicular to the boundary vanishes. This equation \eqref{eqn3} means at boundary the force is 0. Please note that for Neumann problems, the potential at boundary points is arbitrary as long as \eqref{eqn4} is satisfied.

\section{PP 374}
\begin{tcolorbox}
Consider the reflection of $\xi$ according to two boundaries.
\end{tcolorbox}
We have 
\begin{equation}
	\Delta g = \delta(x - \xi), g|_{\partial \Omega} = 0
\end{equation}
So 
\begin{equation}
	E(x,\xi) = \frac{1}{2 \pi} \ln |x- \xi|
\end{equation}
and the sequence of images are 
\begin{equation*}
	\xi_n^{\pm} = (\xi_1, \pm \xi_2 \pm 2 n \pi)
\end{equation*}

Then we have 
\begin{align}
	g(x;\xi) &= \sum_{n \in \mathbb{Z}} E(x, \xi_n^+) - \sum_{n \in \mathbb{Z}} E(x, \xi_n^-) \\
	&= \frac{1}{2 \pi} \sum_{n \in \mathbb{Z}} \ln |x- \xi_n^+| -\frac{1}{2 \pi} \sum_{n \in \mathbb{Z}} \ln |x- \xi_n^-| 
\end{align}
\begin{tcolorbox}[colback=blue!20!white]
Two comments:
\begin{enumerate}
	\item We ignore convergence issue
	\item Introduce complex numbers:
	\begin{align}
	\left|\begin{pmatrix}
			x_1 \\ x_2
		\end{pmatrix}\right| = |x_1+ ix_2| = \sqrt{x_1^2+x_2^2}
	\end{align}
\end{enumerate}
\end{tcolorbox}
Then we could write 
\begin{align*}
	|x- \xi_n^+| = \left| \begin{pmatrix}
		x_1 \\ x_2
	\end{pmatrix} - \begin{pmatrix}
		\xi_1 \\ \xi_2 + 2n \pi
	\end{pmatrix}\right| 
	& = |(x_1 -\xi_1) + (x_2- \xi_2 + 2 n \pi)i| \\
	& = |x - \xi + 2 n \pi i| \quad x = x_1 + x_2 i
\end{align*}
Similarly,
\[
	|x - \xi_n^-| = |x - \bar{\xi} + 2n \pi i ||
\]
SO we have 
\begin{align}
	\sum_{n \in \mathbb{Z}} \ln |x - \xi_n^+| = \sum_{n \in \mathbb{Z}} \ln |x - \xi+ 2n \pi i| = \ln \left( \prod _{n \in \mathbb{Z} } |x- \xi + 2n \pi i|\right)
\end{align}
Now 
\begin{align}
	g(x , \xi) & = \frac{1}{ 2 \pi} \ln \left( \prod _{n \in \mathbb{Z} }  \frac{|x- \xi + 2n \pi i|}{ |x- \bar{ \xi}+ 2n \pi i|}\right) \\
	&= \frac{1}{2 \pi} \ln \left| \left(\prod _{n \in \mathbb{Z} } \frac{\frac{x -\xi}{2 i n \pi}-1 }{\frac{x - \bar{\xi}}{2 n \pi i}-1}\right) \cdot \frac{x - \xi}{x - \bar\xi} \right|
\end{align}
\end{document}















